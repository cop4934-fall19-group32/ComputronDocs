\subsection{Overview}
The purpose of this project is to provide a bridge for potential computer
scientists to learn and explore the fundamentals of solving problems
computationally. Providing this insight is non-trivial, especially in
traditional teaching environments. Our approach is to present common
introductory computer science concepts in the form of puzzles and visuals in a
2D video game. This game will provide a means for those interested in
computational thinking to learn in a fun and engaging environment. The players
will not be expected to set up a development environment or learn the nuances of
specific programming languages. All of the concepts necessary to solving the
challenges should be taught to -- or discovered by -- the player through interactive
game play. Framing this tool as a game is an important distinction, as it allows
us to deliberately craft the players’ experience and deliver specific learning
outcomes.\\

Currently, there are many game offerings that focus on teaching computational
thinking, with most of them focusing on using “code” solutions for puzzle
solving. However, we have found that they tend to fall far on either end of a
spectrum; either the game is too abstract and the computational thinking aspect
is not well emphasized, or the game is too technical and therefore inaccessible
to beginner level programmers. It is our intention to provide a game that falls
between these two extremes on the spectrum. We want our game to teach the user
about the different commands and how they can be used to solve the puzzles
provided. This will lean away from some of the more abstract puzzle-solvers,
where the movements manipulate a character spatially instead of using the
sequence of commands to solve the puzzle. Conversely, we will use clear and
simple language coupled with visual and audio cues to explain commands and data
structures in the game, while abstracting the intricacies of language specific
commands or complex details about these elements. This will help new programmers
to more easily understand the concepts of Computer Science and use them to solve
the puzzles without requiring them to be familiar with language specific syntax
or the complexities of managing data structures in code. For example, we can
provide a heap data structure for them to use and explain its general purpose
for sorting and storing data without forcing the user to construct or maintain
the heap. Taking away the technical aspects of utilizing these elements will
allow players to focus on learning and understanding the concepts of the tools
they are given and how to utilize those tools to solve the problem presented.


\subsection{Broader Impacts}

The audience that this game is geared towards is people with very little prior
technical experience relating to computer science, who are just beginning to
learn how to program and learn computational logic. The aim of this project is
to create an engaging platform for players to explore the process of approaching
and solving problems computationally, build a foundation for understanding the
principles of Computer Science, and prepare users for challenges they may face
in their future studies of Computer Science.\\

At UCF, the difficulties that surround new Computer Science students are easily
seen in the courses leading up to the Foundation Exam, perhaps the most obvious
with Intro to C. Students in that class are immediately faced with two major
tasks: learn to solve problems programmatically, and learn the structure and
syntax of the C programming language. Students often enter the class without
any experience with either of these tasks, and many students end up feeling
lost and discouraged early on.\\

Another aspect that should be considered is the unfortunate number of
underrepresented groups in STEM fields, especially Computer Science. There are
many factors that can contribute to this issue, such as individual backgrounds,
ethnicity, gender, social stigmas, disabilities, or the common lack of
preparation from underfunded public schools. Students who are already facing
pressures related to these issues may be more inclined to give up on Computer
Science when they feel lost in their first year.\\

As with any group of technically proficient individuals, it is not uncommon to
find that a number of the more experienced members inadvertently reinforce a
“gatekeeping” mindset towards STEM majors by supporting weed out classes and
survival of the fittest. This behavior can occur with upper level Computer
Science students or graduates who have seen the large number of highly
intelligent students who fail or give up on Computer Science, which could often
be avoided if better education and resources were widely available.\\

For situations like any of those described above, a resource that can provide
the tools necessary to understand and approach the concepts and challenges of
Computer Science in an effective manner would be an invaluable resource for many
individuals. Our project will provide these tools in an engaging format,
enabling players of different backgrounds and experience levels to cultivate an
intuitive understanding of computational thinking, while giving them a platform
to apply their newfound perspective to approach and solve various problems. As a
result, we hope that our game and similar resources help more students of all
backgrounds to successfully pursue their passion for Computer Science.\\

While the game strives to close the bridge between logic and syntax, by focusing
on the former it can benefit a broader scope of people, regardless of if they
plan to become software developers, or simply wish to learn the logic of
computer systems. The puzzles can act as a means to spark users’ interest in
computer science, or just to be enjoyed by a spectrum of developers and
engineers (students, industry, etc.), as the levels scale in complexity. Introducing
a variety of puzzles that are solved using similar methods and mechanics should
help players to begin recognizing similar patterns between different problems.\\

The game will have an engaging and easy to use interface to provide educational
value with puzzle-based challenges. The mechanics of the game will be easy to
pick up, since learning to play the game should not feel like a difficult task
on its own. The scaling of puzzle difficulty will allow the game to be welcoming
to beginners by incrementally introducing higher complexity tools and abilities
as the levels progress. By providing feedback on the basic runtime and space
complexities of working solutions, players will have the ability to learn how to
solve problems efficiently.\\

\subsection{Personal Motivations}
\subsubsection{Brandy King}
With career goals of being an Instructor of Computer Science, I was immediately
intrigued during this project proposal. As a Teaching Assistant, I have spent a
lot of time working with students in introductory level Computer Science
courses. It is often a difficult subject for students who come in with little to
no experience, and I have watched many students struggle with learning a new
programming language at the same time that they are learning to use that
language to effectively solve problems that they have been presented. I loved
the idea of an interactive game that would not only help teach people some of
the basics, but also reinforce efficient problem solving with computational
logic. I hope that the finished project we aim to provide will help students
entering computer science classes by giving them confidence in their knowledge
of concepts.\\

I am also very excited to learn new technologies, work with a team, and
experiment with game development. As it currently stands, most of my experience
with software development has been strictly course driven. Balancing full time
studies, a full-time work schedule, and a family and personal life leaves me
with extremely limited free time to pursue personal projects or devote time to
learning things outside of lectures. Working on this project will give me the
opportunity to expand my knowledge in a new programming language and to work
with a game engine. I have always been enthusiastic about video games, and I see
their value as an engaging form of media. Being able to use this platform to
convey instruction in Computer Science is a marriage of two things I am very
passionate about.

\subsubsection{John Billingham}
Coming from having no programming experience prior to college, I have seen and
experienced the disconnect between learning the correct syntax of a programming
language, and being able to think and solve problems programmatically. This game
has an aim to teach basic logic and computational thinking skills to it’s users
and I believe that this can close that specified gap that many of us have felt.

\subsubsection{Nicolas LaCognata}
I pitched this project, and I care a great deal about it. I centered my
undergraduate studies around games and simulations, and I’m excited to apply my
knowledge to create a game that is both highly engaging and educational. There
have been many occasions where my desire to pursue a career in game development
has been viewed negatively, and I’d like to be able to point to this project as
proof that the skills I have learned can provide a valuable experience to
others.\\

The idea for this project mainly came from my experience as a Teaching Assistant for Intro to
C. When I took the class, I had prior programming experience. For much of the class I
checked out mentally. I drifted through the class without really seeing what was going on.
Going back and seeing the challenges new programmers really face was a profound
experience. I want this game to help aspiring programmers, and spark interest in
our field for those on the outskirts.

\subsubsection{Sean Simonian}
Like many computer science students, I had very little programming experience
and next to zero understanding of most computer science concepts when I first
started at UCF, and I quickly found myself struggling immensely with the
introductory computer science courses. The main reason I was able to get through
the first year of computer science courses was because of friends who were
willing to help tutor me in the areas where the lectures were not effective
enough at explaining concepts to beginners. In later years, I became a much
better computer science student, and found myself in the opposite role as
before. I have had the opportunity to serve as a mentor to other students
struggling with computer science, and often tutored several people whose
professors’ teaching styles failed to meet the needs of the students. By
experiencing both sides of these situations, I have learned quite a bit about
what beginners struggle with when learning computer science concepts, and how
these concepts need to be broken down and taught in relatable ways. I am excited
about the opportunity to create a tool that can help individuals build a strong
foundation for understanding computer science.\\

Video game development was the primary reason I decided to study computer
science, which motivated me to learn to create a video game for my high school
senior project, and joined the Game Dev Knights club at UCF during my freshman
year. Later I moved away from the goal of game development in favor of interests
in software development and cyber security, so this project will provide the
opportunity to more thoroughly explore game development in a real setting with a
team without having to change the direction of my studies or career path.\\

My technical experience includes working as a software engineer for Harris
Corporation supporting programs for the Federal Aviation Administration, and
later working as a software developer for Texas Instruments supporting their
internal Data Lifecycle Management project. I have also completed a minor in
Secure Computing and Networks, have been an active member of UCF’s Collegiate
Cyber Defense Club for the past few years, and am currently part of the
operations team for Hack@UCF to support the club. Each of these experiences has
helped me build various skills that should be useful for this project. Working
on this game will provide the opportunity to develop additional skills in areas
where I would otherwise be lacking.

