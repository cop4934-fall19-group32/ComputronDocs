This section of the document describes the major milestones of our project. It
is highly susceptible to change.

\subsection{Big Prototype Party (Monday, Oct 14th)}

\subsubsection*{Summary}
Since this project does not have a sponsor, the onus is on us to determine the
exact nature of the problem we would like to solve. To ensure that our efforts
are directed effectively for the remainder of our project, our first month has
been dedicated to researching and playtesting. By this date, every member of the
team should have a solid foundation of the problem space we are working in, and
have a prototype of a solution.

\subsubsection*{Deliverables}
Every member of the team must have a prototype of our final game. This prototype
should fit the following criteria:
\begin{itemize}
  \item The prototype is interactive.
  \item The prototype is engaging to work with.
  \item The prototype represent a wide slice of the game’s mechanics.
  \item The prototype makes the player think computationally.
  \item The rules of the prototype are concise and written out.
\end{itemize}

\subsection{Game Pitch (Friday, October 18th)}

\subsubsection*{Summary}
This milestone culminates one of the most critical stages of our project. Now
that we’ve had the time to demonstrate non-abstractly what our problem space is
and how we’d like to solve it, we can converge on a unified solution. By this
milestone the team should be able to confidently answer the following questions,
and develop the documentation necessary to capture our answers:
\begin{enumerate}
  \item What will our final product look like?
  \item What are the learning objectives of our game?
  \item What kind of puzzles will our game have?
  \item How will those puzzles work?
  \item How will those puzzles achieve our learning objectives?
  \item How will we order these puzzles to ensure that there is an appropriate
  learning curve?
  \item What evidence do we have that our learning curve will be engaging and
  effective for our target audience?
  \item What games did we draw inspiration from?
\end{enumerate}

\subsubsection*{Deliverables}
This milestone’s deliverables are still non-code items. These are items that we
will rely on for the remainder of our project, and inform all future efforts.
Specifically, we must deliver:
\begin{itemize}
  \item A unified prototype of our final game
  \item A write up of our prototype, its capabilities, and its design philosophy
  \item Diagrams/concept drawings of game mechanics not demonstrated by the prototype
  \item Research Papers that support our design choices
  \item Playtesting findings that support our design choices
  \item Any other artifacts necessary to answer the questions above
\end{itemize}

\subsection{Paper Prototype 2.0 (Monday, October 28th)}

\subsubsection*{Summary}
Rapid iteration is an important part of successfully designing a game. By this
milestone, our original prototype should be placed in the hands of as many
playtesters as possible. The group should use the data from this playtesting to
inform any design changes we make to our prototype.

\subsubsection*{Deliverables}
\begin{itemize}
  \item A new prototype of our final game, with documentation supporting our
  changes to our original prototype.
  \item Expanded design documentation to capture the vision of our final game
  including:
  \begin{itemize}
    \item Detailed system diagrams that describe how the mechanics of our game
    interact
    \item An explanation of who will be the lead on certain areas of our game
  \end{itemize}
\end{itemize}

\subsection{Hello Game (Friday, Dec 13th)}

\subsubsection*{Summary}
This milestone marks our the delivery of our proof of concept prototype. The teams advisor will evaluate the prototype and provide reccommendations for moving forward. The prototype will show an extremely basic vertical slice of our final game. 

\subsubsection*{Deliverables}
The only deliverable for this milestone will be the Hello World prototype. This prototype should meet the following requirements:

\begin{itemize}
  \item Usability
  
  The prototype shall show the entire scene flow of our game. A user should be able to move between the Main Menu, Puzzle Selection, and Puzzle scenes at will. The game should respond to these requests gracefully, and never leave the user trapped in a scene. 

  \item Game Mechanics

  The foundations of the game's major mechanics should be in place and interactable. The expected functionalities of the demo are specified below by scene.
  \begin{itemize}
    \item Main Menu
    
    The Main Menu scene should display the game's working title \textbf{\textit{Computron}}, as well as a functional Play Button that will bring the user to the Puzzle Selection Scene. The scene should also have a placeholder background image.

    \item Puzzle Selection Scene
    
    The Puzzle Selection Scene should display a placeholder level graph, with selectable nodes. Each node should transfer the player to the Puzzle Scene. The player should also be able to return to the main menu.

    \begin{figure}[!hb]
      \begin{center}
        \includegraphics[width=5cm]{HelloWorldLevelSelect.png}
        \caption{Sample level graph.}
        \label{fig:boat1}
      \end{center}
    \end{figure}

    \item Puzzle Scene
    
    The Puzzle Scene will be the most involved portion of our prototype. Desired mechanics are listed below by discipline:\\

    \textbf{UI:} The Puzzle UI will need to allow the player to perform the basic puzzle solving operations. It should present the player with an instruction set (\textit{Input, Output}) from which they may drag instructions into a solution window. The Puzzle UI should present controls to allow the player to begin and terminate the simulation of their solution. When the player's solution is being executed, the UI should prohibit alterations to the solution window and display a program counter to mark the current instruction being executed.\\

    \textbf{Level Design:} The level design of the Puzzle Scene will present all of the static elements that will be present in every puzzle of our game. It will display an input box with randomly generated contents, an output box that can accept objects from the actor, and a simple background plate.\\

    \textbf{Puzzle Logic:} Working in tandem with the level design, the puzzle scene's logic system should be operational. Puzzle Logic is responsible for loading the scene with the appropriate puzzle information. This includes a randomly generated input box, the level prompt, an expected output, and the logic to read the output box upon program completion to grade the player's solution.\\
    
    \textbf{Interpreter:} The interpreter is the glue that binds the UI controls to the Actor system. For this milestone, the Interpreter should be able to access and process the player's solution stored in the solution window. When the player starts the simulation the Interpreter should process the instructions (ensuring none are malformed) and make them available to the actor upon request. When the actor requests an instruction, the interpreter will update it's program counter accordingly and report the counter's new position to the UI. The Interpreters PC calculation should be fully operational for the Input, Output, and Unconditional Jump commands\\

    \textbf{Actor:} For this milestone, the actor should be able to query the interpreter for instructions to execute and interact with static level elements. Upon receiving an input command, the Actor should move to the input box and remove at item. The currently held item should be displayed on the Actor. Upon receiving an output command, the Actor should move to the output box and deposit the currently held item. If the actor does not have an item in hand, it should report a fault to the interpreter so that simulation can be halted. If the actor see's two input commands back to back, it should discard its currently held item and attempt to take from the input box again.\\
    
  \end{itemize}

  \item Art
  
  For this milestone programmer art placeholders are acceptable. However, the team should have a plan in place to aquire quality art assets.

\end{itemize}

\subsection{Waterfall Method (Thursday, December 5th)}

\subsubsection*{Summary}
This milestone marks the deadline for Senior Design 1 documentation submission. This document must be complete, professionally bound, and delivered to HEC-345 by 1:00PM.

\subsubsection*{Deliverables}
As per the requirements of this course, our final design document will describe
all aspects of the project including:
\begin{enumerate}
  \item An executive summary
  
  The Executive Summary will give a high level overview about the purpose of the project, and how the project will fulfill that purpose. The executive summary should be written in language accessible to non-technical professionals.

  \item A section detailing project signifigance
  
  This section should echo our executive summary in greater detail, with more specific language.

  \item A section describing our requirements
  
  This section should include an exhaustive list of our project's functional and non-functional requirements.

  \item A section detailing our game's design process
  
  This section should describe the team's efforts in discovering project requirements, and relate the game mechanics chosen to the projects goals.

  \item A section detailing our game's technical design
  
  This section should describe the layout of the game's code systems. It should include detailed technical discussions and diagrams to guide development.

  \item The project's milestones
  
  The milestones will be sets of requirements to ensure the project remains on track.

\end{enumerate}


\subsection{Open to Interpretation (Wednesday, January 15th)}

\subsubsection*{Summary}
"Open to Interpretation" will mark the project's first Alpha release. In this state, the game should be functional enough to place in front of a playtester with minor developer intervention. All basic game mechanics will be fully functional, and observing gameplay should produce relevant observations for the development team.

\subsubsection*{Deliverables}
This milestone’s deliverable is a stable Alpha release of our game. The requirements of the Alpha release are as follows:

\begin{itemize}
  \item Player experience
  
  The player experience should be polished enough to allow an individual with no experience with our project to pick up and play it. All controls presented to the player should work as expected, free of any noticable bugs, or glitches. Menus should allow for proper flow between the game's scenes, and incomplete or non-functional menus should be marked as such.

  Specifically, the player should be able to perform the following operations:

  \begin{enumerate}
    \item Launch the game from outside the Unity Editor
    \item Start the game from the main menu, and enter the puzzle selection scene
    \item From the puzzle selection scene, the player should be able to choose from a limited set of distinct puzzles that represent a vertical slice of our final game's mechanics
    \item After selecting a puzzle, the player should be transfered to the puzzle scene.
    \item In the puzzle scene, the player should be presented with controls relevant to the puzzle selected
    \begin{itemize}
      \item The instruction set pane shows only the subset of instructions chosen to be available for that puzzle.
      \item If the puzzle includes Memory Cards, the cards are displayed at the bottom of the screen and are interactable.
      \item Move and Copy instructions are not usable until at least one Memory card is in play
      \item Memory Cards can be placed on, removed from, and re-odered in the play area.
      \item A Memory Card cannot be removed if it is referenced by an instruction in the solution window. Instructions blocking card removal are highlighted when the player's request fails.
    \end{itemize}
  \end{enumerate}


  \item Available Puzzles
  
  Puzzles 1, 2a, 3, and 5 of the tutorial sequence specified in Section~\textbf{\ref{section:tutorial}}

  \item Available Mechanics
  \begin{itemize}
    \item Level Selection
    
    The level selection scene allows players to chose which puzzle they'd like to attempt. Chosing a level should transfer the player to the Puzzle Scene with the relevant puzzle data loaded. The player will not be restricted in which puzzles are available, or which order they chose to complete them.

    \item Basic instruction writing
    
    Player can click and drag the INPUT, OUTPUT, MOVETO, MOVEFROM, JUMP, and JUMP IF NULL instructions into the instruction pane. Jump instructions allow player to click-and-drag to set jump anchor.

    \item Basic Memory Card Manipulation
    
    Player is presented and can interact with a hand of memory cards on levels that require them. Cards can be played, reorganized, and removed. MOVETO and MOVEFROM instructions are active if and only if there is at least one Memory Card on the board.

    \item Basic Solution Grading
    
    After a player completes a puzzle, they are presented with a score sheet that provides metrics on the par for:
    \begin{itemize}
      \item Instruction count
      \item Solution Runtime
      \item Memory Card cost
    \end{itemize}
  \end{itemize}
\end{itemize}

\subsection{Tutorializing (Wednesday, February 12th)}

\subsubsection*{Summary}
"Tutorializing" will mark the Computron's second Alpha release. For this release, the tutorial system will be greatly expanded to allow for more hands-off playtesting.

\subsubsection*{Deliverables}
This milestone’s deliverable is another stable Alpha release of our game. The expected new features are listed below:

\begin{itemize}

  \item Available Puzzles
  
  Puzzles 1 through 9 of the tutorial sequence specified in Section~\textbf{\ref{section:tutorial}}

  \item Available Mechanics
  \begin{itemize}
    \item Level Selection
    The level selection menu should meet the following requirements:
    \begin{itemize}
      \item Level selection menu supports a strong ordering of tutorial levels
      \item Level selection menu allows for optional paths that do not impede player progression
      \item The player's performace on puzzles (number of stars) should be recorded and restored between play sessions.
      \item Advanced levels should be restricted from play until the player earns the resequite number of points to unlock them.
    \end{itemize}

    \item Instruction Writing
    
    The puzzle writing and interpreting system should support the following instructions:

    \begin{enumerate}
      \item INPUT
      \item OUTPUT
      \item JUMP
      \item JUMP IF NULL
      \item JUMP IF LESS
      \item JUMP IF GREATER
      \item MOVETO X
      \item MOVEFROM x
    \end{enumerate}

    \item Memory Card Manipulation
    
    Player memory card interactions should be well polished. Players should find playing and removing cards an intuitive part of the puzzle solving process. In addition, the following cards should be operational:

    \begin{enumerate}
      \item Register
      \item Stack
    \end{enumerate}
  \end{itemize}
\end{itemize}

\subsection{Breaking Beta (Friday, February 28th)}

\subsubsection*{Summary}
"Breaking Beta" is Computron's first Beta release. At this point, the game's minimum viable product should be feature complete, leaving ample time to rework and refine mechanics based on intense playtesting.

\subsubsection*{Deliverables}
\begin{itemize}

  \item Available Puzzles
  
  Puzzles 1 through 12 of the tutorial sequence specified in Section~\textbf{\ref{section:tutorial}}

  \item Available Mechanics
  \begin{itemize}
    \item Instruction Introductions
    
    Messaging should be added to attract the player's attention to new instructions when they are made available. Upon first receipt of a new instruction, the User Interface should display an interactive menu that allows players to unbox their new command.

    \item Instruction Writing
    
    In game documentation of how instructions work is available from the puzzle scene

    \item Memory Card Manipulation
    
    In game documentation of how a memory card works is available from the puzzle scene
  \end{itemize}
\end{itemize}

\subsection{FIEA Playtest (Monday, March 2nd)}
\subsubsection*{Summary}
The team's advisor has offered to allow Computron to be playtested by students at the Florida Interactive Entertainnment Acedemy (FIEA). Playtesters from FIEA will span both technical and non-technical roles in game development, and will be able to offer valuable insights. It will be imperative to capture these insights, and use them to guide the final stages of the game's development

\subsubsection*{Deliverables}
For this playtest, the team should be able to present a high quality build of the game. Fielding a high quality demo will be very important in allowing the team to gather effective observations from the playtesters. The team should arrive to the playtest with:

\begin{enumerate}
  \item A stable build of the game
  \item A set of exit survey questions to track player experience
  \item Several computers capable of playing the game to allow increased playtesting bandwidth
\end{enumerate}

After the playtest, the team should compile player feedback and evaluate which parts of the game will need more attention.

\subsection{Quick Math (Friday, March 20th)}

\subsubsection*{Summary}
"Quick Math" will be the second Beta release of Computron. It will feature an expanded instruction set and new puzzles that make use of mathematical principles. 

\subsubsection*{Deliverables}
\begin{itemize}

  \item New Instructions
  Computron should support 4 brand new instructions:
  \begin{enumerate}
    \item ADD X
    \item SUBTRACT X
    \item BUMP++
    \item BUMP--
  \end{enumerate}
  
  \item New Puzzles
  This release should feature 3 new puzzles that exercise the new problem spaces made accessible by the expanded instruction set:

  \begin{enumerate}
    \item Adding Machine
    
    This puzzle should introduce players to the mechanics of the new ADD instruction. Puzzle should ask players to add pairs of numbers in the input and output their sum.

    \item Subractinator

    This puzzle should introduce players to the mechanics of the new SUBTRACT instruction. Puzzle should ask players to subtract pairs of numbers in the input and output their difference.

    \item Hexiplier

    This puzzle should show players an application of the new ADD instruction. For each number in the input, the puzzle should expect that number times 6 as output.

  \end{enumerate}

\end{itemize}

\subsection{Release Candidate (Friday, April 10th)}
\subsubsection*{Summary}
Nearing the end of our project, milestone requirements will largely be determined by the state of the game and the results of playtests. For this milestone, the team should have most of the concerns raised from the FIEA playtest fully remedied.

\subsubsection*{Deliverables}
A fun release candidate of Computron.

\subsection{Mission Accomplished (Wednesday, April 15th)}
\subsubsection*{Summary}
"Mission Accomplished" marks the final release of Computron. The project should be in a state that accomplishes the goals set forth in this document.

\subsubsection*{Deliverables}
A game the team is proud of.
