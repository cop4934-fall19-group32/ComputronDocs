The tutorial sequence of \textit{Computron} will introduce players to the game mechanics and allow players to acclimate to the puzzle solving workflow. The ordering of the tutorial sequence is crafted to control the rate at which players learn, and provide a smooth difficulty curve while constantly presenting new mechanics and concepts to keep player interest high.\\

\paragraph{Level 1: Ins and Outs}
\subparagraph{Learning Objective:} Teach players the very basics of our user interface through a simple puzzle.

\subparagraph{Problem:} There's only a single input to output here. I hope you can handle it!

\begin{center}
    \begin{tabular}{ | m{6cm} | m{8cm} | } 
        \hline
            \textbf{Available Instructions:} 
            \begin{itemize}
                \setlength\itemsep{-.35em}
                \item INPUT
                \item OUTPUT
            \end{itemize}& 
            \textbf{Expected Solution:} 
            \begin{enumerate}
                \setlength\itemsep{-.35em}
                \item INPUT
                \item OUTPUT
            \end{enumerate}
            \\
        \hline
    \end{tabular}
\end{center}


\paragraph{Level 2: Ins and Outs pt. 2}
\subparagraph{Learning Objective:} Ensure that the player understands the basic mechanics of commanding Computron by requiring them to chain the previous puzzle's solution. In addition, prepare the player to understand the value of loops.

\subparagraph{Problem:} Oh boy! Can you do that again five whole times?

\begin{center}
    \begin{tabular}{ | m{6cm} | m{8cm} | } 
        \hline
            \textbf{Available Instructions:} 
            \begin{itemize}
                \setlength\itemsep{-.35em}
                \item INPUT
                \item OUTPUT
            \end{itemize}& 
            \textbf{Expected Solution:} 
            \begin{enumerate}
                \setlength\itemsep{-.35em}
                \item INPUT
                \item OUTPUT
                \item INPUT
                \item OUTPUT
                \item INPUT
                \item OUTPUT
                \item INPUT
                \item OUTPUT
                \item INPUT
                \item OUTPUT
            \end{enumerate}
            \\
        \hline
    \end{tabular}
\end{center}


\paragraph{Level 2a (Optional): Alternator}
\subparagraph{Learning Objective:} Demonstrate a nuanced mechanic of the input instruction. It is not apparent from the previous puzzles, but executing two input commands in succession will cause Computron to discard the first.

\subparagraph{Problem:} I don't like the first and third number. Just give me the other two.

\begin{center}
    \begin{tabular}{ | m{6cm} | m{8cm} | } 
        \hline
            \textbf{Available Instructions:} 
            \begin{itemize}
                \setlength\itemsep{-.35em}
                \item INPUT
                \item OUTPUT
            \end{itemize}& 
            \textbf{Expected Solution:} 
            \begin{enumerate}
                \setlength\itemsep{-.35em}
                \item INPUT
                \item INPUT
                \item OUTPUT
                \item INPUT
                \item INPUT
                \item OUTPUT
            \end{enumerate}
            \\
        \hline
    \end{tabular}
\end{center}


\paragraph{Level 2b (Optional): Triplicate}
\subparagraph{Learning Objective:} Have players incorporate the idea of skipping over inputs in conjunction with the power of using loops.

\subparagraph{Problem:} I only want every third item. If you give me anything else, I'll cry. Well, the Computron equivalent of crying.

\begin{center}
    \begin{tabular}{ | m{6cm} | m{8cm} | } 
        \hline
            \textbf{Available Instructions:} 
            \begin{itemize}
                \setlength\itemsep{-.35em}
                \item INPUT
                \item OUTPUT
		\item JUMP
		\item JUMP IF NULL
            \end{itemize}& 
            \textbf{Expected Solution:} 
            \begin{enumerate}
                \setlength\itemsep{-.35em}
                \item INPUT
		\item JUMP IF NULL [EOP]
		\item INPUT
		\item INPUT
                \item OUTPUT
		\item JUMP 1
            \end{enumerate}
            \\
        \hline
    \end{tabular}
\end{center}


\paragraph{Level 3: Hip Hop}
\subparagraph{Learning Objective:} Introduce the concept of jumping and looping.

\subparagraph{Problem:} Someone dumped a bunch of garbage in my input! Throw it all in the output, I guess.

\begin{center}
    \begin{tabular}{ | m{6cm} | m{8cm} | } 
        \hline
            \textbf{Available Instructions:} 
            \begin{itemize}
                \setlength\itemsep{-.35em}
                \item INPUT
                \item OUTPUT
                \item JUMP
                \item JUMP IF NULL
            \end{itemize}& 
            \textbf{Expected Solution:} 
            \begin{enumerate}
                \setlength\itemsep{-.35em}
                \item INPUT
                \item JUMP IF NULL [EOP]
                \item OUTPUT
                \item JUMP 1
            \end{enumerate}
            \\
        \hline
    \end{tabular}
\end{center}


\paragraph{Level 4: Flip Flop}
\subparagraph{Learning Objective:} Introduce variable declaration, and present the player with their first significant challenge.

\subparagraph{Problem:} I want both of these items, but I don't like the order. Give them to me in reverse.

\begin{center}
    \begin{tabular}{ | m{6cm} | m{8cm} | } 
        \hline
            \textbf{Available Instructions:} 
            \begin{itemize}
                \setlength\itemsep{-.35em}
                \item INPUT
                \item OUTPUT
                \item JUMP
                \item JUMP IF NULL
                \item MOVE TO X
                \item MOVE FROM X
            \end{itemize}
            \textbf{Available Cards:} 
            \begin{itemize}
                \setlength\itemsep{-.35em}
                \item Register x2
            \end{itemize}& 
            \textbf{Expected Solution:} 
            \begin{enumerate}
                \setlength\itemsep{-.35em}
                \item Play Register [0]
                \item INPUT
                \item MOVE TO [0]
                \item INPUT
                \item OUTPUT
                \item MOVE FROM [0] 
                \item OUTPUT
            \end{enumerate}
            \\
        \hline
    \end{tabular}
\end{center}


\paragraph{Level 5: Flip Flop Hop}
\subparagraph{Learning Objective:} Allow players to experiment with the new register mechanic, and guide them to generalize their previous solution.

\subparagraph{Problem:} Keep the pairs together, but give them to me in reverse. Just like before but, y'know, more times.

\begin{center}
    \begin{tabular}{ | m{6cm} | m{8cm} | } 
        \hline
            \textbf{Available Instructions:} 
            \begin{itemize}
                \setlength\itemsep{-.35em}
                \item INPUT
                \item OUTPUT
                \item JUMP
                \item JUMP IF NULL
                \item MOVE TO X
                \item MOVE FROM X
            \end{itemize}
            \textbf{Available Cards:} 
            \begin{itemize}
                \setlength\itemsep{-.35em}
                \item Register x2
            \end{itemize}& 
            \textbf{Expected Solution:} 
            \begin{enumerate}
                \setlength\itemsep{-.35em}
                \item Play Register [0]
                \item INPUT
                \item JUMP IF NULL [EOP]
                \item MOVE TO [0]
                \item INPUT
                \item OUTPUT
                \item MOVE FROM [0] 
                \item OUTPUT
                \item JUMP 2
            \end{enumerate}
            \\
        \hline
    \end{tabular}
\end{center}


\paragraph{Level 6: Triceraflops}
\subparagraph{Learning Objective:} Expand player's understanding of the power of registers, and start guiding them toward seeing the value of stacks.

\subparagraph{Problem:} Since you've got pairs down, wanna try it with triples? You better.

\begin{center}
    \begin{tabular}{ | m{6cm} | m{8cm} | } 
        \hline
            \textbf{Available Instructions:} 
            \begin{itemize}
                \setlength\itemsep{-.35em}
                \item INPUT
                \item OUTPUT
                \item JUMP
                \item JUMP IF NULL
                \item MOVE TO X
                \item MOVE FROM X
            \end{itemize}
            \textbf{Available Cards:} 
            \begin{itemize}
                \setlength\itemsep{-.35em}
                \item Register x4
            \end{itemize}& 
            \textbf{Expected Solution:} 
            \begin{enumerate}
                \setlength\itemsep{-.35em}
                \item Play Register [0]
                \item Play Register [1]
                \item INPUT
                \item MOVE TO [0]
                \item INPUT
                \item MOVE TO [1]
                \item INPUT
                \item OUTPUT
                \item MOVE FROM [1] 
                \item OUTPUT
                \item MOVE FROM [0] 
                \item OUTPUT
            \end{enumerate}
            \\
        \hline
    \end{tabular}
\end{center}


\paragraph{Level 6a (Optional): Triceraflops Hops}
\subparagraph{Learning Objective:} Force the player to generalize their previous solution for an arbitrary number of inputs.

\subparagraph{Problem:} So you reversed a triple once, but can you do it again and again?

\begin{center}
    \begin{tabular}{ | m{6cm} | m{8cm} | } 
        \hline
            \textbf{Available Instructions:} 
            \begin{itemize}
                \setlength\itemsep{-.35em}
                \item INPUT
                \item OUTPUT
                \item JUMP
                \item JUMP IF NULL
                \item MOVE TO X
                \item MOVE FROM X
            \end{itemize}
            \textbf{Available Cards:} 
            \begin{itemize}
                \setlength\itemsep{-.35em}
                \item Register x4
            \end{itemize}& 
            \textbf{Expected Solution:} 
            \begin{enumerate}
                \setlength\itemsep{-.35em}
                \item Play Register [0]
                \item Play Register [1]
                \item INPUT
                \item JUMP IF NULL [EOP]
                \item MOVE TO [0]
                \item INPUT
                \item MOVE TO [1]
                \item INPUT
                \item OUTPUT
                \item MOVE FROM [1] 
                \item OUTPUT
                \item MOVE FROM [0]
                \item OUTPUT
                \item JUMP 3
            \end{enumerate}
            \\
        \hline
    \end{tabular}
\end{center}


\paragraph{Level 6b (Optional): Quanta-Flop}
\subparagraph{Learning Objective:} Move towards the idea of using multiple registers to track the data as a cumbersome task, reinforce the incoming usefulness of a stack.

\subparagraph{Problem:} These four numbers are all backwards! Place them into the output box in ascending order.

\begin{center}
    \begin{tabular}{ | m{6cm} | m{8cm} | } 
        \hline
            \textbf{Available Instructions:} 
            \begin{itemize}
                \setlength\itemsep{-.35em}
                \item INPUT
                \item OUTPUT
                \item JUMP
                \item JUMP IF NULL
                \item MOVE TO X
                \item MOVE FROM X
            \end{itemize}
            \textbf{Available Cards:} 
            \begin{itemize}
                \setlength\itemsep{-.35em}
                \item Register x4
            \end{itemize}& 
            \textbf{Expected Solution:} 
            \begin{enumerate}
                \setlength\itemsep{-.35em}
                \item Play Register [0]
                \item Play Register [1]
		\item Play Register [2]
                \item INPUT
                \item MOVE TO [0]
                \item INPUT
                \item MOVE TO [1]
                \item INPUT
		\item MOVE TO [2]
		\item INPUT
                \item OUTPUT
		\item MOVE FROM [2]
		\item OUTPUT
                \item MOVE FROM [1] 
                \item OUTPUT
                \item MOVE FROM [0] 
                \item OUTPUT
            \end{enumerate}
            \\
        \hline
    \end{tabular}
\end{center}


\paragraph{Level 7: Fat Stacks}
\subparagraph{Learning Objective:} Introduce players to the Stack data structure, and demonstrate its FILO property by having them reverse the input stream.

\subparagraph{Problem:} There's a lot of inputs here. Can you put them all to the output in reverse order?

\begin{center}
    \begin{tabular}{ | m{6cm} | m{8cm} | } 
        \hline
            \textbf{Available Instructions:} 
            \begin{itemize}
                \setlength\itemsep{-.35em}
                \item INPUT
                \item OUTPUT
                \item JUMP
                \item JUMP IF NULL
                \item MOVE TO X
                \item MOVE FROM X
            \end{itemize}
            \textbf{Available Cards:} 
            \begin{itemize}
                \setlength\itemsep{-.35em}
                \item Register x5
                \item Stack x2
            \end{itemize}& 
            \textbf{Expected Solution:} 
            \begin{enumerate}
                \setlength\itemsep{-.35em}
                \item Play Stack [0]
                \item INPUT
                \item JUMP IF NULL [6]
                \item MOVE TO [0]
                \item JUMP 2
                \item MOVE FROM [0]
                \item JUMP IF NULL [EOP]
                \item OUTPUT
                \item JUMP 6
            \end{enumerate}
            \\
        \hline
    \end{tabular}
\end{center}


\paragraph{Level 8: Comparatron}
\subparagraph{Learning Objective:} Introduce players to conditional jumps, and show that jumps can be used to select branches of logic.

\subparagraph{Problem:} I only want the greatest - of every pair of items, that is.

\begin{center}
    \begin{tabular}{ | m{6cm} | m{8cm} | } 
        \hline
            \textbf{Available Instructions:} 
            \begin{itemize}
                \setlength\itemsep{-.35em}
                \item INPUT
                \item OUTPUT
                \item JUMP
                \item JUMP IF NULL
                \item JUMP IF LESS X
                \item JUMP IF GREATER X
		\item JUMP IF EQUAL X
                \item MOVE TO X
                \item MOVE FROM X
            \end{itemize}
            \textbf{Available Cards:} 
            \begin{itemize}
                \setlength\itemsep{-.35em}
                \item Register x5
                \item Stack x2
            \end{itemize}& 
            \textbf{Expected Solution:} 
            \begin{enumerate}
                \setlength\itemsep{-.35em}
                \item Play Register [0]
                \item INPUT
	     \item JUMP IF NULL [EOP]
                \item MOVE TO [0]
                \item INPUT
                \item JUMP IF GREATER [0], 8
                \item MOVE FROM [0]
                \item OUTPUT
	     \item JUMP 1
            \end{enumerate}
            \\
        \hline
    \end{tabular}
\end{center}


\paragraph{Level 9: Big Fish}
\subparagraph{Learning Objective:} Allows players to generalize the concept of maximization shown in the previous problem to a more robust situation, and continue making them comfortable with using jumps to select branches of logic.

\subparagraph{Problem:} Give me the biggest number you can find!

\begin{center}
    \begin{tabular}{ | m{6cm} | m{8cm} | } 
        \hline
            \textbf{Available Instructions:} 
            \begin{itemize}
                \setlength\itemsep{-.35em}
                \item INPUT
                \item OUTPUT
                \item JUMP
                \item JUMP IF NULL
                \item JUMP IF LESS X
                \item JUMP IF GREATER X
	     \item JUMP IF EQUAL X
                \item MOVE TO X
                \item MOVE FROM X
            \end{itemize}
            \textbf{Available Cards:} 
            \begin{itemize}
                \setlength\itemsep{-.35em}
                \item Register x5
                \item Stack x2
            \end{itemize}& 
            \textbf{Expected Solution:} 
            \begin{enumerate}
                \setlength\itemsep{-.35em}
                \item Play Register [0]
                \item INPUT
                \item MOVE TO [0]
                \item INPUT
                \item JUMP IF NULL 9
                \item JUMP IF LESS [0], 4
                \item MOVE TO [0]
                \item JUMP 4
                \item MOVE FROM [0]
                \item OUTPUT
            \end{enumerate}
            \\
        \hline
    \end{tabular}
\end{center}


\paragraph{Level 9a (Optional): Two Fish}
\subparagraph{Learning Objective:} Allows players to challenge themselves with a more difficult variant of the previous problem.

\subparagraph{Problem:} I'm greedy, give me the two biggest numbers this time.

\begin{center}
    \begin{tabular}{ | m{6cm} | m{8cm} | } 
        \hline
            \textbf{Available Instructions:} 
            \begin{itemize}
                \setlength\itemsep{-.35em}
                \item INPUT
                \item OUTPUT
                \item JUMP
                \item JUMP IF NULL
                \item JUMP IF LESS X
                \item JUMP IF GREATER X
		\item JUMP IF EQUAL X
                \item MOVE TO X
                \item MOVE FROM X
            \end{itemize}
            \textbf{Available Cards:} 
            \begin{itemize}
                \setlength\itemsep{-.35em}
                \item Register x5
                \item Stack x2
            \end{itemize}& 
            \textbf{Expected Solution:} 
            \begin{enumerate}
                \setlength\itemsep{-.35em}
                \item Play Register [0]
                \item Play Register [1]
                \item Play Register [2]
                \item INPUT
                \item MOVE TO [0]
                \item INPUT
                \item MOVE TO [1]
                \item INPUT
                \item JUMP IF NULL 21
                \item JUMP IF LESS [0], 18
                \item MOVE TO [2]
                \item MOVE FROM [0]
		\item JUMP IF LESS [1], 15
                \item MOVE TO [1]
                \item MOVE FROM [2]
                \item MOVE TO [0]
                \item JUMP 8
                \item JUMP IF LESS [1], 8
                \item MOVE TO [1]
                \item JUMP 8
                \item MOVE FROM [0]
		\item JUMP IF GREATER [1], 28
		\item MOVE TO [2]
		\item MOVE FROM [1]
		\item OUTPUT
		\item MOVE FROM [2]
		\item JUMP 30
                \item OUTPUT
                \item MOVEFROM [1]
                \item OUTPUT
            \end{enumerate}
            \\
        \hline
    \end{tabular}
\end{center}


\paragraph{Level 9b (Optional): Red Fish}
\subparagraph{Learning Objective:} Originally, this level was designed to extend the concepts of the previous level. However, upon implementation in the game it was found that the solution was actually deeply difficult at this stage in the process, so the requirements were changed for it to be the last level available and it can only be unlocked if every other star in the game is earned. When all of the abilities of the game are unlocked, the challenge becomes much simpler and the goal becomes testing the player to put their resources together and find the simplest solution.

\subparagraph{Problem:} Give me the three largest numbers in the input.

\begin{center}
    \begin{tabular}{ | m{6cm} | m{8cm} | } 
        \hline
            \textbf{Available Instructions:} 
            \begin{itemize}
                \setlength\itemsep{-.35em}
                \item INPUT
                \item OUTPUT
                \item JUMP
                \item JUMP IF NULL
                \item JUMP IF LESS X
                \item JUMP IF GREATER X
	     \item JUMP IF EQUAL X
                \item MOVE TO X
                \item MOVE FROM X
            \end{itemize}
            \textbf{Available Cards:} 
            \begin{itemize}
                \setlength\itemsep{-.35em}
                \item Register x5
                \item Stack x2
		\item Queue x2
		\item Heap x2
            \end{itemize}& 
            \textbf{Expected Solution:} 
            \begin{enumerate}
                \setlength\itemsep{-.60em}
                \item Play Heap [0]
                \item INPUT
                \item JUMP IF NULL 6
                \item MOVETO [0]
                \item JUMP 2
                \item MOVEFROM [0]
                \item OUTPUT
                \item MOVEFROM [0]
                \item OUTPUT
                \item MOVEFROM [0]
                \item OUTPUT
            \end{enumerate}
            \\
        \hline
    \end{tabular}
\end{center}


\paragraph{Level 10: Little Fish}
\subparagraph{Learning Objective:} Encourage players to explore the concept of minimization.

\subparagraph{Problem:} I feel bad for only caring about big inputs. Give me the smallest one you can find this time.

\begin{center}
    \begin{tabular}{ | m{6cm} | m{8cm} | } 
        \hline
            \textbf{Available Instructions:} 
            \begin{itemize}
                \setlength\itemsep{-.35em}
                \item INPUT
                \item OUTPUT
                \item JUMP
                \item JUMP IF NULL
                \item JUMP IF LESS X
                \item JUMP IF GREATER X
		\item JUMP IF EQUAL X
                \item MOVE TO X
                \item MOVE FROM X
            \end{itemize}
            \textbf{Available Cards:} 
            \begin{itemize}
                \setlength\itemsep{-.35em}
                \item Register x5
                \item Stack x2
            \end{itemize}& 
            \textbf{Expected Solution:} 
            \begin{enumerate}
                \setlength\itemsep{-.35em}
                \item Play Register [0]
                \item INPUT
                \item MOVETO [0]
                \item INPUT
                \item JUMP IF NULL 9
                \item JUMP IF GREATER [0], 8
                \item MOVE TO [0]
                \item JUMP 4
                \item MOVE FROM [0]
                \item OUTPUT
            \end{enumerate}
            \\
        \hline
    \end{tabular}
\end{center}


\paragraph{Level 11: Equal Fish}
\subparagraph{Learning Objective:} Encourage players to explore the concept of equality.

\subparagraph{Problem:} I really like that first number. Give me every one of them that you find.

\begin{center}
    \begin{tabular}{ | m{6cm} | m{8cm} | } 
        \hline
            \textbf{Available Instructions:} 
            \begin{itemize}
                \setlength\itemsep{-.35em}
                \item INPUT
                \item OUTPUT
                \item JUMP
                \item JUMP IF NULL
                \item JUMP IF LESS X
                \item JUMP IF GREATER X
		\item JUMP IF EQUAL X
                \item MOVE TO X
                \item MOVE FROM X
            \end{itemize}
            \textbf{Available Cards:} 
            \begin{itemize}
                \setlength\itemsep{-.35em}
                \item Register x5
                \item Stack x2
            \end{itemize}& 
            \textbf{Expected Solution:} 
            \begin{enumerate}
                \setlength\itemsep{-.35em}
                \item Play Register [0]
                \item INPUT
                \item MOVETO [0]
                \item INPUT
                \item JUMP IF NULL 10
                \item JUMP IF EQUAL [0], 8
                \item JUMP 4
                \item OUTPUT
		\item JUMP 4
                \item MOVE FROM [0]
                \item OUTPUT
            \end{enumerate}
            \\
        \hline
    \end{tabular}
\end{center}


\paragraph{Level 12: Heap Fish}
\subparagraph{Learning Objective:} Introduce players to their second data structure, the Heap.

\subparagraph{Problem:} Now I want it all - give me everything you find, and do it from largest to smallest.

\begin{center}
    \begin{tabular}{ | m{6cm} | m{8cm} | } 
        \hline
            \textbf{Available Instructions:} 
            \begin{itemize}
                \setlength\itemsep{-.35em}
                \item INPUT
                \item OUTPUT
                \item JUMP
                \item JUMP IF NULL
                \item JUMP IF LESS X
                \item JUMP IF GREATER X
	     \item JUMP IF EQUAL X
                \item MOVE TO X
                \item MOVE FROM X
            \end{itemize}
            \textbf{Available Cards:} 
            \begin{itemize}
                \setlength\itemsep{-.35em}
                \item Register x5
                \item Stack x2
                \item Heap x2
            \end{itemize}& 
            \textbf{Expected Solution:} 
            \begin{enumerate}
                \setlength\itemsep{-.35em}
                \item Play Heap [0]
                \item INPUT
                \item JUMP IF NULL 6
                \item MOVETO [0]
                \item JUMP 2
                \item MOVEFROM [0]
                \item JUMP IF NULL [EOP]
                \item OUTPUT
		\item JUMP 6
            \end{enumerate}
            \\
        \hline
    \end{tabular}
\end{center}


\paragraph{Level 13: Reverse Fish}
\subparagraph{Learning Objective:} Show players the power of combining data structures to accomplish a new task.

\subparagraph{Problem:} Scratch that - reverse it. Give them all to me, smallest to largest.

\begin{center}
    \begin{tabular}{ | m{6cm} | m{8cm} | } 
        \hline
            \textbf{Available Instructions:} 
            \begin{itemize}
                \setlength\itemsep{-.35em}
                \item INPUT
                \item OUTPUT
                \item JUMP
                \item JUMP IF NULL
                \item JUMP IF LESS X
                \item JUMP IF GREATER X
		\item JUMP IF EQUAL X
                \item MOVE TO X
                \item MOVE FROM X
            \end{itemize}
            \textbf{Available Cards:} 
            \begin{itemize}
                \setlength\itemsep{-.35em}
                \item Register x5
                \item Stack x2
                \item Heap x2
            \end{itemize}& 
            \textbf{Expected Solution:} 
            \begin{enumerate}
                \setlength\itemsep{-.35em}
                \item Play Heap [0]
                \item Play Stack [1]
                \item INPUT
                \item JUMP IF NULL 7
                \item MOVE TO [0]
                \item JUMP 3
                \item MOVE FROM [0]
                \item JUMP IF NULL 11
                \item MOVE TO [1]
                \item JUMP 7
                \item MOVE FROM [1]
                \item JUMP IF NULL [EOP]
                \item OUTPUT
                \item JUMP 11
            \end{enumerate}
            \\
        \hline
    \end{tabular}
\end{center}


\paragraph{Level 14: Pivot Fish}
\subparagraph{Learning Objective:} Introduce players to the Queue data structure and highlight their usefulness of keeping things in order.

\subparagraph{Problem:} Pivot time! Save the first item and output everything smaller. Then give me the pivot and all the ones that were larger. Keep 'em in the order you find 'em!

\begin{center}
    \begin{tabular}{ | m{6cm} | m{8cm} | } 
        \hline
            \textbf{Available Instructions:} 
            \begin{itemize}
                \setlength\itemsep{-.35em}
                \item INPUT
                \item OUTPUT
                \item JUMP
                \item JUMP IF NULL
                \item JUMP IF LESS X
                \item JUMP IF GREATER X
		\item JUMP IF EQUAL X
                \item MOVE TO X
                \item MOVE FROM X
            \end{itemize}
            \textbf{Available Cards:} 
            \begin{itemize}
                \setlength\itemsep{-.35em}
                \item Register x5
		\item Stack x2
                \item Queue x2
                \item Heap x2
            \end{itemize}& 
            \textbf{Expected Solution:} 
            \begin{enumerate}
                \setlength\itemsep{-.35em}
		\item Play Register [0]
                \item Play Queue [1]
                \item INPUT
		\item MOVE TO [0]
		\item INPUT
                \item JUMP IF NULL 12
		\item JUMP IF LESS [0], 10
                \item MOVE TO [1]
                \item JUMP 5
		\item OUTPUT
		\item JUMP 5
                \item MOVE FROM [0]
		\item OUTPUT
		\item MOVE FROM [1]
                \item JUMP IF NULL [EOP]
                \item OUTPUT
                \item JUMP 14
            \end{enumerate}
            \\
        \hline
    \end{tabular}
\end{center}


\paragraph{Level 15: Copy Constructor}
\subparagraph{Learning Objective:} Introduce players to the copy to and copy from instructions, allowing them to make duplicates of data values.

\subparagraph{Problem:} I like this number so much -- I want two of them!

\begin{center}
    \begin{tabular}{ | m{6cm} | m{8cm} | } 
        \hline
            \textbf{Available Instructions:} 
            \begin{itemize}
                \setlength\itemsep{-.35em}
                \item INPUT
                \item OUTPUT
                \item JUMP
                \item JUMP IF NULL
                \item JUMP IF LESS X
                \item JUMP IF GREATER X
		\item JUMP IF EQUAL X
                \item MOVE TO X
                \item MOVE FROM X
                \item COPY TO X
                \item COPY FROM X
            \end{itemize}
            \textbf{Available Cards:} 
            \begin{itemize}
                \setlength\itemsep{-.35em}
                \item Register x5
		\item Stack x2
                \item Queue x2
                \item Heap x2
            \end{itemize}& 
            \textbf{Expected Solution:} 
            \begin{enumerate}
                \setlength\itemsep{-.35em}
		\item Play Register [0]
                \item INPUT
		\item COPY TO [0]
		\item OUTPUT
                \item COPY FROM [0]
		\item OUTPUT
            \end{enumerate}
            \\
        \hline
    \end{tabular}
\end{center}


\paragraph{Level 16: ALU}
\subparagraph{Learning Objective:} Introduce players to the add instruction, allowing them to perform addition and, by extension, multiplication on data values.

\subparagraph{Problem:} These numbers are good, but they could be twice as good. Give me the double of each!

\begin{center}
    \begin{tabular}{ | m{6cm} | m{8cm} | } 
        \hline
            \textbf{Available Instructions:} 
            \begin{itemize}
                \setlength\itemsep{-.35em}
                \item INPUT
                \item OUTPUT
                \item JUMP
                \item JUMP IF NULL
                \item JUMP IF LESS X
                \item JUMP IF GREATER X
		\item JUMP IF EQUAL X
                \item MOVE TO X
                \item MOVE FROM X
                \item COPY TO X
                \item COPY FROM X
		\item ADD X
            \end{itemize}
            \textbf{Available Cards:} 
            \begin{itemize}
                \setlength\itemsep{-.35em}
                \item Register x5
		\item Stack x2
                \item Queue x2
                \item Heap x2
            \end{itemize}& 
            \textbf{Expected Solution:} 
            \begin{enumerate}
                \setlength\itemsep{-.35em}
		\item Play Register [0]
                \item INPUT
		\item JUMP IF NULL [EOP]
		\item COPY TO [0]
		\item ADD [0]
		\item OUTPUT
		\item JUMP 2
            \end{enumerate}
            \\
        \hline
    \end{tabular}
\end{center}


\paragraph{Level 17: Mega-Multiplier}
\subparagraph{Learning Objective:} Encourage players to achieve multiplication through repeated addition.

\subparagraph{Problem:} Why stop at double? Let's make it the quadruple of each number!

\begin{center}
    \begin{tabular}{ | m{6cm} | m{8cm} | } 
        \hline
            \textbf{Available Instructions:} 
            \begin{itemize}
                \setlength\itemsep{-.35em}
                \item INPUT
                \item OUTPUT
                \item JUMP
                \item JUMP IF NULL
                \item JUMP IF LESS X
                \item JUMP IF GREATER X
		\item JUMP IF EQUAL X
                \item MOVE TO X
                \item MOVE FROM X
                \item COPY TO X
                \item COPY FROM X
		\item ADD X
            \end{itemize}
            \textbf{Available Cards:} 
            \begin{itemize}
                \setlength\itemsep{-.35em}
                \item Register x5
		\item Stack x2
                \item Queue x2
                \item Heap x2
            \end{itemize}& 
            \textbf{Expected Solution:} 
            \begin{enumerate}
                \setlength\itemsep{-.35em}
		\item Play Register [0]
                \item INPUT
		\item JUMP IF NULL [EOP]
		\item COPY TO [0]
		\item ADD [0]
		\item COPY TO [0]
		\item ADD [0]
		\item OUTPUT
		\item JUMP 2
            \end{enumerate}
            \\
        \hline
    \end{tabular}
\end{center}


\paragraph{Level 18: Ultra-Multiplier}
\subparagraph{Learning Objective:} Encourage players to achieve exponential multiplication through repeated addition and storage of the results.

\subparagraph{Problem:} I want MORE! Give me the number times sixteen this time.

\begin{center}
    \begin{tabular}{ | m{6cm} | m{8cm} | } 
        \hline
            \textbf{Available Instructions:} 
            \begin{itemize}
                \setlength\itemsep{-.35em}
                \item INPUT
                \item OUTPUT
                \item JUMP
                \item JUMP IF NULL
                \item JUMP IF LESS X
                \item JUMP IF GREATER X
		\item JUMP IF EQUAL X
                \item MOVE TO X
                \item MOVE FROM X
                \item COPY TO X
                \item COPY FROM X
		\item ADD X
            \end{itemize}
            \textbf{Available Cards:} 
            \begin{itemize}
                \setlength\itemsep{-.35em}
                \item Register x5
		\item Stack x2
                \item Queue x2
                \item Heap x2
            \end{itemize}& 
            \textbf{Expected Solution:} 
            \begin{enumerate}
                \setlength\itemsep{-.35em}
		\item Play Register [0]
                \item INPUT
		\item JUMP IF NULL [EOP]
		\item COPY TO [0]
		\item ADD [0]
		\item COPY TO [0]
		\item ADD [0]
		\item COPY TO [0]
		\item ADD [0]
		\item COPY TO [0]
		\item ADD [0]
		\item OUTPUT
		\item JUMP 2
            \end{enumerate}
            \\
        \hline
    \end{tabular}
\end{center}


\paragraph{Level 19: Mister Negative}
\subparagraph{Learning Objective:} Introduce players to the subtract command and encourage them to explore the idea of negation.

\subparagraph{Problem:} I ate too much and need to balance out. For every number here, give me its negative equivalent.

\begin{center}
    \begin{tabular}{ | m{6cm} | m{8cm} | } 
        \hline
            \textbf{Available Instructions:} 
            \begin{itemize}
                \setlength\itemsep{-.35em}
                \item INPUT
                \item OUTPUT
                \item JUMP
                \item JUMP IF NULL
                \item JUMP IF LESS X
                \item JUMP IF GREATER X
		\item JUMP IF EQUAL X
                \item MOVE TO X
                \item MOVE FROM X
                \item COPY TO X
                \item COPY FROM X
		\item ADD X
		\item SUBTRACT X
            \end{itemize}
            \textbf{Available Cards:} 
            \begin{itemize}
                \setlength\itemsep{-.35em}
                \item Register x5
		\item Stack x2
                \item Queue x2
                \item Heap x2
            \end{itemize}& 
            \textbf{Expected Solution:} 
            \begin{enumerate}
                \setlength\itemsep{-.35em}
		\item Play Register [0]
                \item INPUT
		\item JUMP IF NULL [EOP]
		\item COPY TO [0]
		\item SUBTRACT [0]
		\item SUBTRACT [0]
		\item OUTPUT
		\item JUMP 2
            \end{enumerate}
            \\
        \hline
    \end{tabular}
\end{center}