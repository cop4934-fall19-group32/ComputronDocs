\subsubsection{Ideas}
The beginning of our project was characterized by a great deal of brainstorming. We took great strides
to avoid a herd mentality when discussing possible game mechanics, themes, and designs. We frequently
separated to come up with ideas as individuals before returning to blend those ideas together into a more
cohesive whole.

\paragraph{Brandy King}
\begin{itemize}
  \item Smaller puzzles can be the parts of a larger cohesive piece
  \begin{itemize}
    \item The efficiency of a solution can dictate how well each piece works
    together
    \item Will reinforce revisiting puzzles that have already been solved to
    find more efficient solutions (leading to a better solution overall)
  \end{itemize}
  \item Variable and Data Structure Declaring
  \begin{itemize}
    \item Separate declaration field from the solution space
    \item Can add or remove variables/structures dynamically
  \end{itemize}
  \item Tutorial levels
  \begin{itemize}
    \item Explaining new instructions or data structures in a non-puzzle level
    \item Allow users to “free play” with their available commands and see how
    things work without the pressure of needing to use them to solve a puzzle
    \item A “to-do” list with checkboxes of key concepts they need for each
    command
    \begin{itemize}
      \item E.G. A Heap tutorial level “to-do” list would include:
      \begin{itemize}
        \item Declare a Heap
        \item Put five inputs in the Heap
        \item Remove five inputs from the Heap
      \end{itemize}
    \end{itemize}
    \item The “to-do” list must be completed before the user can progress to the
    next puzzle, where they will need to utilize the new element in the solution
  \end{itemize}
  \item Points-based scoring of solutions
  \begin{itemize}
    \item Quantify the solution provided with a ranking
    \item Informs users on the efficiency of their solution
  \end{itemize}
  \item Limiting certain commands
  \begin{itemize}
    \item Can’t select to use a data structure space that isn’t declared
    \item Certain commands will not be applicable to data structures
    \begin{itemize}
      \item I.e., data stored in a structure must be removed to be manipulated;
      cannot arbitrarily add or subtract values stored there
    \end{itemize}
    \item Commands with changeable parts will be limited to available locations
  \end{itemize}
  \item Indentation within if statements to emphasize scope
  \begin{itemize}
    \item Will more clearly define to users which commands are reached within
    conditionals; especially important when nested conditionals are introduced
  \end{itemize}
  \item Instruction “Jumbotron”
  \begin{itemize}
    \item An area above the puzzle space that broadcasts which command is being
    executed at that time
  \end{itemize}
\end{itemize}

\paragraph{John Billingham}
\begin{itemize}
  \item Abstracting conditional jumps with more conventional IF/ELSE statements
  \begin{itemize}
    \item Instructions can be nested under condional statements. This is meant to
    cause less confusion and more visual aid to a player who may not be familiar
    with the idea of jumping to another place in the instructions. We must be careful
    and not allow the player to create complex nested logic by limiting the number
    of conditional instructions they can use.
  \end{itemize}
  \item Counter that shows how many jumps have been used
  \begin{itemize}
    \item This could enforce the idea of runtime complexity. More jumps means more
    looping which could slow down a player's solution. A reward system could also
    be introduced that awards points based on how fast the solution was. Jump
    instructions could be given as a finite resource to also enforce and teach the
    idea of runtime efficiency.
  \end{itemize}
  \item Data Structures as Input/Output
  \begin{itemize}
    \item Arrays, Stack, Heaps, Queues etc
    \item Different data structures are unlocked as the puzzle complexity increases.
    \item A differnt instruction set is associated with each data structure, abstracting
    some of the earlier instructions as the puzzle's progress.
  \end{itemize}
  \item Atomic move instructions
  \begin{itemize}
    \item Replacing need for an actor to pick up and place down data  in different places by combing those into
    one atomic "Move" instruction. This is meant to simplify things for the player by keeping
    the previous functionality but removing an unneeded intermediate step of work. Data would have
    a source and a destination when using this instruction, without needing to go through
    the original actor mechanism.
  \end{itemize}
  \item Add/Subtract specificities
  \begin{itemize}
    \item Drop-down menus that allow you to choose two source locations to sum, and
    a destination. This is also meant to simplify the instructions for the player.
  \end{itemize}
  \item Color-based indexing system
  \begin{itemize}
    \item The player accesses instructions through a color-based indexing system.
    Colors represent specific data structures on the game board, while indices represent
    specific locations within a colored data structure. "Green sub 3" would indicate the
    3rd index within the green data structure.
  \end{itemize}
  \item Return instruction
  \begin{itemize}
    \item Allow the user to choose a data structure to return as their output. The
    contents of the data structure are checked agaisnt the expected output for the
    puzzle level. The player chooses the data structure they wish to return by choosing
    that data structure's color in the output game object.
  \end{itemize}
  \item Debugging levels
  \begin{itemize}
    \item The player can step back and forth through their solution, one instruction at a time,
    and the actor should support these operations to show a visual debugging process. This
    visual control can help players find problems within their solution, allowing them to
    progressively reach the end of puzzle level.
  \end{itemize}
  \item Broken solution given initially
  \begin{itemize}
    \item Player must fix a broken solution that is given when the puzzle level starts.
    \item Can be used in conjunction with tutorials to teach new concepts. Enforces the fuctionality
    of certain data structures by having the palyer interact with a broken data structure being
    introduced to them.
  \end{itemize}
\end{itemize}

\paragraph{Nicolas LaCognata:}\mbox{} \\
My biggest inspiration for this project was \textit{Human Resource Machine} by the Tomorrow Corporation.
\textit{Human Resource Machine} is a charming little puzzle game that tackles the problem we settled on solving,
the gamification of computational thinking. I love the game, and find it highly engaging. However, playtests
with non-programmers clearly demonstrated the games shortcomings when dealing with its target audience.

\subparagraph{Basic Mechanics:}\mbox{} \\
When brainstorming the basic mechanics of our game, my mind continually strayed towards \textit{Human Resource Machine}.
Through playing the game, and watching others play it, it became clear that Tomorrow Corporation's decision to stick with
an assembly-like instruction set was an extremely effective mechanic.\\

The assembly like instructions were simple enough for non-programmers to understand, and easy to compose together to make more
advanced puzzles and solutions. Even at the risk of having our game appear as a rip-off, I thought it would be wise to adapt
these mechanics.

\subparagraph{Data Structures:}\mbox{} \\
After surveying many games in the problem space, we failed to find an example of a game that tackles computational thinking
with the aid of Data Structures. Data Structures are such a fundamental part of being a programmer, that their omission in these
games seemed incorrect.

\subparagraph{Card Game Mechanics:}\mbox{} \\
During our prototyping sessions, I came up with the idea of “Memory Cards”.
These memory cards would allow the player to declare register locations and data
structures before their program starts running. Making the mechanics of placing these
memory cards similar to a game like \textit{Hearthstone: Heroes of Warcraft} would be
a natural and compelling addition to our puzzle format. This “Memory Card” interface
would be separate from the normal instruction writing interface, which helps reinforce
the idea that they are separate parts of the puzzle. \\

Another benefit of this design is the ability to deliver new instructions and datastructures
to the player in a familiar (and exceptionally game-y) way, card packs. Card games deliver
goodies to players in little packages whose contents appear hidden. While discussing adapting
card game mechanics, we realized that having that little unboxing moment when introducing new
instructions to the player could be highly compelling. Other games in our problem space often
silently drop new instructions on the player as they progress. By making the new instructions
appear as a reward, we can ensure that the new mechanic being introduced gets the appropriate
attention from the player.

\paragraph{Sean Simonian}
\begin{itemize}
  \item Static commands and restricted dynamic commands
  \begin{itemize}
    \item Certain commands in the language should be static, such as “Read
    Input”
    \item Certain commands in the language should have components that can be
    selected from drop down menus, such as “Jump if (blank) (blank) (blank)”
    \begin{itemize}
      \item This command would have 3 sections to specify, so the user could
      specify this command to be “Jump if x > 0”
    \end{itemize}
  \end{itemize}
  \item Break more complex challenges into multiple steps that get checked along
  the way.
  \begin{itemize}
    \item User would have instructions to complete one step of the complete
    level challenge at a time, run their solution, and if they pass then they
    move onto the next step, etc., until they work up to the solution.
    \item This would help players grasp more complex concepts that would be
    harder to teach as a single challenge that they have to break down on their
    own.
    \item Example: Merge Sort algorithm
    \begin{itemize}
      \item Step 1: split an array in half
      \item Step 2: split an array in half recursively
      \item Step 3: compare 2 of the sub arrays and swap them if necessary
      \item Step 4: recursively compare and swap sub arrays to build back up to
      the full sorted array
    \end{itemize}
  \end{itemize}
  \item For testing the educational efficacy of the game, we could create a set
  of short questions to be done on paper that require computational thinking.
  Before a test subject plays our game, we have them work through a few of the
  questions. Then after the test subject plays our game, we have them work
  through a few more questions from the set. Compare the results to see if they
  indicate whether our game helps users with computational thinking and problem
  solving.
  \item For testing the educational efficacy of the game, conducting official
  tests through the UCF psych department would be very beneficial if we are able
  to do so.
  \begin{itemize}
    \item Students taking courses such as Intro to Psychology are required to
    spend a certain number of hours as a test subject as part of their course
    grade. They use SONA to view available tests and sign up for a time slot,
    and the experiments that involve playing video games tend to fill up very
    quickly.
    \item Since Intro to Psychology is a general education course, many subjects
    are freshman, and the majority are not studying computer science or similar
    majors. This could provide an abundant set of candidates that match our
    target audience.
    \item We have to figure out who to contact if we want to set this up for
    next semester.
    \item The UCF psychology department may only allow these official tests to
    be set up and run by psych majors and/or psych graduate students doing
    research. If this is the case, we could look for a grad student conducting
    educational research who would be interested in “sponsoring” these official
    tests.
  \end{itemize}
  \item Incorporate basic cyber security and cryptography concepts into additional
  challenges/puzzles.
  \begin{itemize}
    \item This was a goal early on, but the current design for the game
    mechanics would make this difficult to add to the game.
  \end{itemize}
  \item Initial ideas for computer science concepts to teach. As we narrowed
  down the specifications for our game, these ideas were thrown out as they will
  not fit in the game at all
  \begin{itemize}
    \item Artificial intelligence, neural networks, machine learning
    \item Blockchain technology and its applications
  \end{itemize}
\end{itemize}
