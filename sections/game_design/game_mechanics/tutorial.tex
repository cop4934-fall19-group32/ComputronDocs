The tutorial sequence of \textit{Computron} will introduce players to the game's mechanics and allow players to acclimate to the puzzle solving workflow. The ordering of the tutorial sequence is crafted to control the rate at which players learn, and provide a smooth difficulty curve while constantly presenting new mechanics and concepts. \\

\paragraph{Level 1: Ins and Outs}
\subparagraph{Learning Objective:} Teach players the very basics of our user interface through a simple puzzle.

\subparagraph{Problem:} There is a single item in the input box. Please transfer it into the output box.

\subparagraph{Solution:} 
\begin{center}
    \begin{tabular}{ | m{5cm} | m{9cm} | } 
        \hline
            \textbf{Available Instructions:} 
            \begin{itemize}
                \setlength\itemsep{-.35em}
                \item INPUT
                \item OUTPUT
            \end{itemize}& 
            \textbf{Expected Solution:} 
            \begin{enumerate}
                \setlength\itemsep{-.35em}
                \item INPUT
                \item OUTPUT
            \end{enumerate}
            \\
        \hline
    \end{tabular}
\end{center}


\paragraph{Level 2: Ins and Outs pt. 2}
\subparagraph{Learning Objective:} Ensure that the player understands the basic mechanics of commanding Computron by requiring them to chain the previous puzzle's solution. In addition, prepare the player to understand the value of loops.

\subparagraph{Problem:} There are five items in the input box. Please transfer all of them into the output box.

\subparagraph{Solution:} 
\begin{center}
    \begin{tabular}{ | m{5cm} | m{9cm} | } 
        \hline
            \textbf{Available Instructions:} 
            \begin{itemize}
                \setlength\itemsep{-.35em}
                \item INPUT
                \item OUTPUT
            \end{itemize}& 
            \textbf{Expected Solution:} 
            \begin{enumerate}
                \setlength\itemsep{-.35em}
                \item INPUT
                \item OUTPUT
                \item INPUT
                \item OUTPUT
                \item INPUT
                \item OUTPUT
                \item INPUT
                \item OUTPUT
                \item INPUT
                \item OUTPUT
            \end{enumerate}
            \\
        \hline
    \end{tabular}
\end{center}

\paragraph{Level 2a (Optional): Alternator}
\subparagraph{Learning Objective:} Demonstrate a nuanced mechanic of the input instruction. It it not apparent from the previous puzzles, but executing two input commands in succession will cause Computron to discard the first.

\subparagraph{Problem:} There are 6 items in the input box. Please transfer every other item to the output box, starting with the second input.

\subparagraph{Solution:} 
\begin{center}
    \begin{tabular}{ | m{5cm} | m{9cm} | } 
        \hline
            \textbf{Available Instructions:} 
            \begin{itemize}
                \setlength\itemsep{-.35em}
                \item INPUT
                \item OUTPUT
            \end{itemize}& 
            \textbf{Expected Solution:} 
            \begin{enumerate}
                \setlength\itemsep{-.35em}
                \item INPUT
                \item INPUT
                \item OUTPUT
                \item INPUT
                \item INPUT
                \item OUTPUT
                \item INPUT
                \item INPUT
                \item OUTPUT
            \end{enumerate}
            \\
        \hline
    \end{tabular}
\end{center}

\paragraph{Level 3: Hip Hop}
\subparagraph{Learning Objective:} Introduce the concept of jumping and looping.

\subparagraph{Problem:} There are 100 items in the input box. Please transfer each item to the output box, starting with the second input.

\subparagraph{Solution:} 
\begin{center}
    \begin{tabular}{ | m{5cm} | m{9cm} | } 
        \hline
            \textbf{Available Instructions:} 
            \begin{itemize}
                \setlength\itemsep{-.35em}
                \item INPUT
                \item OUTPUT
                \item JUMP
                \item JUMP IF NULL
            \end{itemize}& 
            \textbf{Expected Solution:} 
            \begin{enumerate}
                \setlength\itemsep{-.35em}
                \item INPUT
                \item JUMP IF NULL [EOP]
                \item OUTPUT
                \item JUMP 1
            \end{enumerate}
            \\
        \hline
    \end{tabular}
\end{center}

\paragraph{Level 4: Flip Flop}
\subparagraph{Learning Objective:} Introduce variable declaration, and present the player with their first signifigant challenge.

\subparagraph{Problem:} There are two items in the input box. Place them in the outbox in reverse order.

\subparagraph{Solution:} 
\begin{center}
    \begin{tabular}{ | m{5cm} | m{9cm} | } 
        \hline
            \textbf{Available Instructions:} 
            \begin{itemize}
                \setlength\itemsep{-.35em}
                \item INPUT
                \item OUTPUT
                \item JUMP
                \item JUMP IF NULL
                \item MOVTO X
                \item MOVEFROM X
            \end{itemize}
            \textbf{Available Cards:} 
            \begin{itemize}
                \setlength\itemsep{-.35em}
                \item Register x2
            \end{itemize}& 
            \textbf{Expected Solution:} 
            \begin{enumerate}
                \setlength\itemsep{-.35em}
                \item Play Register [0]
                \item INPUT
                \item MOVETO [0]
                \item INPUT
                \item OUTPUT
                \item MOVEFROM [0] 
                \item OUTPUT
            \end{enumerate}
            \\
        \hline
    \end{tabular}
\end{center}

\paragraph{Level 5: Flip Flop Hop}
\subparagraph{Learning Objective:} Allow players to experiement with the new register mechanic, and guide them to generalize their previous solution.

\subparagraph{Problem:} There are 50 items in the input box. Place each pair of of items in the inbox to the outbox in reverse order.

\subparagraph{Solution:} 
\begin{center}
    \begin{tabular}{ | m{5cm} | m{9cm} | } 
        \hline
            \textbf{Available Instructions:} 
            \begin{itemize}
                \setlength\itemsep{-.35em}
                \item INPUT
                \item OUTPUT
                \item JUMP
                \item JUMP IF NULL
                \item MOVTO X
                \item MOVEFROM X
            \end{itemize}
            \textbf{Available Cards:} 
            \begin{itemize}
                \setlength\itemsep{-.35em}
                \item Register x2
            \end{itemize}& 
            \textbf{Expected Solution:} 
            \begin{enumerate}
                \setlength\itemsep{-.35em}
                \item Play Register [0]
                \item INPUT
                \item JUMP IF NULL [EOP]
                \item MOVETO [0]
                \item INPUT
                \item OUTPUT
                \item MOVEFROM [0] 
                \item OUTPUT
                \item JUMP 2
            \end{enumerate}
            \\
        \hline
    \end{tabular}
\end{center}

\paragraph{Level 6: Triceraflops}
\subparagraph{Learning Objective:} Expand player's understanding of the power of registers, and start guiding them toward seeing the value of stacks.

\subparagraph{Problem:} There are 3 items in the input box. Place all three in the output box in reverse order.

\subparagraph{Solution:} 
\begin{center}
    \begin{tabular}{ | m{5cm} | m{9cm} | } 
        \hline
            \textbf{Available Instructions:} 
            \begin{itemize}
                \setlength\itemsep{-.35em}
                \item INPUT
                \item OUTPUT
                \item JUMP
                \item JUMP IF NULL
                \item MOVTO X
                \item MOVEFROM X
            \end{itemize}
            \textbf{Available Cards:} 
            \begin{itemize}
                \setlength\itemsep{-.35em}
                \item Register x4
            \end{itemize}& 
            \textbf{Expected Solution:} 
            \begin{enumerate}
                \setlength\itemsep{-.35em}
                \item Play Register [0]
                \item Play Register [1]
                \item INPUT
                \item MOVETO [0]
                \item INPUT
                \item MOVETO [1]
                \item INPUT
                \item OUTPUT
                \item MOVEFROM [1] 
                \item OUTPUT
                \item MOVEFROM [0] 
                \item OUTPUT
            \end{enumerate}
            \\
        \hline
    \end{tabular}
\end{center}

\paragraph{Level 7: Triceraflops Hops}
\subparagraph{Learning Objective:} Force the player to generalize their previous solution for N inputs

\subparagraph{Problem:} There are 90 items in the input box. For each set of three in the inbox, transfer to the outbox in reverse order.

\subparagraph{Solution:} 
\begin{center}
    \begin{tabular}{ | m{5cm} | m{9cm} | } 
        \hline
            \textbf{Available Instructions:} 
            \begin{itemize}
                \setlength\itemsep{-.35em}
                \item INPUT
                \item OUTPUT
                \item JUMP
                \item JUMP IF NULL
                \item MOVTO X
                \item MOVEFROM X
            \end{itemize}
            \textbf{Available Cards:} 
            \begin{itemize}
                \setlength\itemsep{-.35em}
                \item Register x4
            \end{itemize}& 
            \textbf{Expected Solution:} 
            \begin{enumerate}
                \setlength\itemsep{-.35em}
                \item Play Register [0]
                \item Play Register [1]
                \item INPUT
                \item JUMP IF NULL [EOP]
                \item MOVETO [0]
                \item INPUT
                \item MOVETO [1]
                \item INPUT
                \item OUTPUT
                \item MOVEFROM [1] 
                \item OUTPUT
                \item MOVEFROM [0]
                \item OUTPUT
                \item JUMP 3
            \end{enumerate}
            \\
        \hline
    \end{tabular}
\end{center}

\paragraph{Level 8: Fat Stacks}
\subparagraph{Learning Objective:} Introduce players to the Stack datastructure, and demonstrate its FIFO property by having them reverse the input stream.

\subparagraph{Problem:} There are 6 items in the input box. Transfer each item into the outbox in reverse order.

\subparagraph{Solution:} 
\begin{center}
    \begin{tabular}{ | m{5cm} | m{9cm} | } 
        \hline
            \textbf{Available Instructions:} 
            \begin{itemize}
                \setlength\itemsep{-.35em}
                \item INPUT
                \item OUTPUT
                \item JUMP
                \item JUMP IF NULL
                \item MOVTO X
                \item MOVEFROM X
            \end{itemize}
            \textbf{Available Cards:} 
            \begin{itemize}
                \setlength\itemsep{-.35em}
                \item Register x4
                \item Stack x1
            \end{itemize}& 
            \textbf{Expected Solution:} 
            \begin{enumerate}
                \setlength\itemsep{-.35em}
                \item Play Stack [0]
                \item INPUT
                \item JUMP IF NULL [6]
                \item MOVETO 0
                \item JUMP 2
                \item MOVEFROM [0]
                \item JUMP IF NULL [EOP]
                \item OUTPUT
                \item JUMP 6
            \end{enumerate}
            \\
        \hline
    \end{tabular}
\end{center}

\paragraph{Level 9: Comparatron}
\subparagraph{Learning Objective:} Introduce players to conditional jumps, and show that jumps can be used to select branches of logic.

\subparagraph{Problem:} The input box has two items in it. output the greater of the two.

\subparagraph{Solution:} 
\begin{center}
    \begin{tabular}{ | m{5cm} | m{9cm} | } 
        \hline
            \textbf{Available Instructions:} 
            \begin{itemize}
                \setlength\itemsep{-.35em}
                \item INPUT
                \item OUTPUT
                \item JUMP
                \item JUMP IF NULL
                \item JUMP IF LESS X
                \item JUMP IF GREATER X
                \item MOVTO X
                \item MOVEFROM X
            \end{itemize}
            \textbf{Available Cards:} 
            \begin{itemize}
                \setlength\itemsep{-.35em}
                \item Register x2
            \end{itemize}& 
            \textbf{Expected Solution:} 
            \begin{enumerate}
                \setlength\itemsep{-.35em}
                \item Play Register [0]
                \item INPUT
                \item MOVETO [0]
                \item INPUT
                \item JUMP IF GREATER [0], 7
                \item MOVEFROM [0]
                \item OUTPUT
            \end{enumerate}
            \\
        \hline
    \end{tabular}
\end{center}

\paragraph{Level 10: Big Fish}
\subparagraph{Learning Objective:} Allows players to generalize concept of maximization shown in the previous problem to a more robust situation, and continue making them confortable with using jumps to select branches of logic.

\subparagraph{Problem:} The input box has some items in it. Output only the greatest number in the inbox.

\subparagraph{Solution:} 
\begin{center}
    \begin{tabular}{ | m{5cm} | m{9cm} | } 
        \hline
            \textbf{Available Instructions:} 
            \begin{itemize}
                \setlength\itemsep{-.35em}
                \item INPUT
                \item OUTPUT
                \item JUMP
                \item JUMP IF NULL
                \item JUMP IF LESS X
                \item JUMP IF GREATER X
                \item MOVTO X
                \item MOVEFROM X
            \end{itemize}
            \textbf{Available Cards:} 
            \begin{itemize}
                \setlength\itemsep{-.35em}
                \item Register x2
            \end{itemize}& 
            \textbf{Expected Solution:} 
            \begin{enumerate}
                \setlength\itemsep{-.35em}
                \item Play Register [0]
                \item INPUT
                \item MOVETO [0]
                \item INPUT
                \item JUMP IF NULL 9
                \item JUMP IF LESS [0], 4
                \item MOVETO [0]
                \item JUMP 4
                \item MOVEFROM [0]
                \item OUTPUT
            \end{enumerate}
            \\
        \hline
    \end{tabular}
\end{center}

\paragraph{Level 10a (Optional): Two Fish}
\subparagraph{Learning Objective:} Allows players to challenge themselves with a more difficult variant of the previous problem

\subparagraph{Problem:} The input box has some items in it. Output only the two largest numbers in ascending order.

\subparagraph{Solution:} 
\begin{center}
    \begin{tabular}{ | m{5cm} | m{9cm} | } 
        \hline
            \textbf{Available Instructions:} 
            \begin{itemize}
                \setlength\itemsep{-.35em}
                \item INPUT
                \item OUTPUT
                \item JUMP
                \item JUMP IF NULL
                \item JUMP IF LESS X
                \item JUMP IF GREATER X
                \item MOVTO X
                \item MOVEFROM X
            \end{itemize}
            \textbf{Available Cards:} 
            \begin{itemize}
                \setlength\itemsep{-.35em}
                \item Register x4
            \end{itemize}& 
            \textbf{Expected Solution:} 
            \begin{enumerate}
                \setlength\itemsep{-.35em}
                \item Play Register [0]
                \item Play Register [1]
                \item Play Register [2]
                \item INPUT
                \item MOVETO [0]
                \item INPUT
                \item MOVETO [1]
                \item INPUT
                \item JUMP IF NULL 20
                \item JUMP IF LESS [0], 17
                \item MOVETO [2]
                \item MOVEFROM [0]
                \item MOVETO [1]
                \item MOVEFROM [2]
                \item MOVETO [0]
                \item JUMP 8
                \item JUMP IF LESS [1], 8
                \item MOVETO [1]
                \item JUMP 8
                \item MOVEFROM [1]
                \item OUTPUT
                \item MOVEFROM [0]
                \item OUTPUT
            \end{enumerate}
            \\
        \hline
    \end{tabular}
\end{center}

\paragraph{Level 10b (Optional): Red Fish}
\subparagraph{Learning Objective:} Allows players to challenge themselves with a more difficult variant of the previous two problems. Lead them to see the value of max heaps.

\subparagraph{Problem:} The input box has some items in it. Output only the three largest numbers in ascending order.

\subparagraph{Solution:} 
\begin{center}
    \begin{tabular}{ | m{5cm} | m{9cm} | } 
        \hline
            \textbf{Available Instructions:} 
            \begin{itemize}
                \setlength\itemsep{-.35em}
                \item INPUT
                \item OUTPUT
                \item JUMP
                \item JUMP IF NULL
                \item JUMP IF LESS X
                \item JUMP IF GREATER X
                \item MOVTO X
                \item MOVEFROM X
            \end{itemize}
            \textbf{Available Cards:} 
            \begin{itemize}
                \setlength\itemsep{-.35em}
                \item Register x4
            \end{itemize}& 
            \textbf{Expected Solution:} 
            \begin{enumerate}
                \setlength\itemsep{-.60em}
                \item Play Register [0]
                \item Play Register [1]
                \item Play Register [2]
                \item Play Register [3]
                \item INPUT
                \item MOVETO [0]
                \item INPUT
                \item MOVETO [1]
                \item INPUT
                \item MOVETO [2]
                \item INPUT
                \item JUMP IF NULL 32
                \item JUMP IF LESS [0], 22
                \item MOVETO [3]
                \item MOVEFROM [1]
                \item MOVETO [2]
                \item MOVEFROM [0]
                \item MOVETO [1]
                \item MOVEFROM [3]
                \item MOVETO [0]
                \item JUMP 11
                \item JUMP IF LESS [1], 28
                \item MOVETO [3]
                \item MOVEFROM [1]
                \item MOVETO [2]
                \item MOVEFROM [3]
                \item MOVETO [1]
                \item JUMP 11
                \item JUMP IF LESS [2], 11
                \item MOVETO 2
                \item JUMP 11
                \item MOVEFROM [2]
                \item OUTPUT
                \item MOVEFROM [1]
                \item OUTPUT
                \item MOVEFROM [0]
                \item OUTPUT
            \end{enumerate}
            \\
        \hline
    \end{tabular}
\end{center}

\paragraph{Level 11: You Fish}
\subparagraph{Learning Objective:} Introduce players to their second datastructure, the heap.

\subparagraph{Problem:} Place all the items in the inbox into the outbox in descending order.

\subparagraph{Solution:} 
\begin{center}
    \begin{tabular}{ | m{5cm} | m{9cm} | } 
        \hline
            \textbf{Available Instructions:} 
            \begin{itemize}
                \setlength\itemsep{-.35em}
                \item INPUT
                \item OUTPUT
                \item JUMP
                \item JUMP IF NULL
                \item JUMP IF LESS X
                \item JUMP IF GREATER X
                \item MOVTO X
                \item MOVEFROM X
            \end{itemize}
            \textbf{Available Cards:} 
            \begin{itemize}
                \setlength\itemsep{-.35em}
                \item Register x4
                \item Max Heap x1
            \end{itemize}& 
            \textbf{Expected Solution:} 
            \begin{enumerate}
                \setlength\itemsep{-.35em}
                \item Play Max Heap [0]
                \item INPUT
                \item JUMP IF NULL 6
                \item MOVETO [0]
                \item JUMP 2
                \item MOVEFROM [0]
                \item JUMP IF NULL [EOP]
                \item OUTPUT
            \end{enumerate}
            \\
        \hline
    \end{tabular}
\end{center}

\paragraph{Level 12: Sad Fish}
\subparagraph{Learning Objective:} Show players the power of combining datastructures to accomplish a new task.

\subparagraph{Problem:} Place all the items in the inbox into the outbox in ascending order.

\subparagraph{Solution:} 
\begin{center}
    \begin{tabular}{ | m{5cm} | m{9cm} | } 
        \hline
            \textbf{Available Instructions:} 
            \begin{itemize}
                \setlength\itemsep{-.35em}
                \item INPUT
                \item OUTPUT
                \item JUMP
                \item JUMP IF NULL
                \item JUMP IF LESS X
                \item JUMP IF GREATER X
                \item MOVTO X
                \item MOVEFROM X
            \end{itemize}
            \textbf{Available Cards:} 
            \begin{itemize}
                \setlength\itemsep{-.35em}
                \item Register x4
                \item Max Heap x1
                \item Stack x1
            \end{itemize}& 
            \textbf{Expected Solution:} 
            \begin{enumerate}
                \setlength\itemsep{-.35em}
                \item Play Max Heap [0]
                \item Play Stack [1]
                \item INPUT
                \item JUMP IF NULL 7
                \item MOVETO [0]
                \item JUMP 3
                \item MOVEFROM [0]
                \item JUMP IF NULL [11]
                \item MOVETO [1]
                \item JUMP [7]
                \item MOVEFROM [1]
                \item JUMP IF NULL [EOP]
                \item OUTPUT
                \item JUMP 11
            \end{enumerate}
            \\
        \hline
    \end{tabular}
\end{center}