For my prototype, I spent a lot of time focusing on how we should integrate data structures into our assembly-based
puzzle structure. The approach I took saw players attaching data structures to registers to ``augment" them and
change how the register stores and processes information. I also took a stab at defining a theme for our game.\\

\textbf{Setting}\\
In \textit{Automata}, you play an Automaton stuck in a laboratory who is tasked with reprogramming itself to solve problems.
When solving the puzzle, they player may be addressed by a narrator when they make a mistake. The game was to have 
a ``Mad Science" feel to it, much like \textit{Portal}.\\

\textbf{Rules}\\
For each puzzle the player is presented with a problem, and they must use the provided instructions to solve it. 
Every puzzle will contain the following elements:
\begin{itemize}
	\item An Input track
	\item An Output track
	\item A set of static registers built into the test chamber
	\begin{itemize}
		\item These registers allow the player to store and manage multiple data cubes
	\end{itemize}
	\item A set of register augmenters
	\begin{itemize}
		\item These augmenters turn the registers into special data structures that can make the puzzles easier/possible
	\end{itemize}
	\item An instruction set
	\begin{itemize}
		\item These are the instructions the Automaton can execute to solve the puzzle; when placed in the solution window,
		they will be executed in the order they appear
		\item Some operations are invalid for certain states of the puzzle. For your safety, please avoid executing invalid operations.
	\end{itemize}
\end{itemize}
\newpage

\textbf{Instruction Set}
\begin{center}
    \begin{tabular}{ | m{3cm} | m{11cm} | } 
        \hline
            \begin{center}
                \textbf{INPUT} 
            \end{center}& 
            \textbf{Summary:} 
            \newline Allows your automata to grab an item from the input track. It places it in its hands.

            \textbf{Notes:} 
            \newline Your automata will drop any item currently in their hands to accept the new input, losing the old data forever.
            \newline If the input track is empty, the automata will do nothing

            \textbf{Faults:}
            \newline None\\
        \hline
            \begin{center}
                \textbf{OUTPUT} 
            \end{center}& 
            \textbf{Summary:} 
            \newline Allows your automata to place the item in its register into the output track.

            \textbf{Notes:} 
            \newline None

            \textbf{Faults:}
            \newline None\\
        \hline
            \begin{center}
                \textbf{SUBMIT} 
            \end{center}& 
            \textbf{Summary:} 
            \newline Allows your automata to submit its output for evaluation.

            \textbf{Notes:} 
            \newline This should always be the last instruction in your program.

            \textbf{Faults:}
            \newline Output Empty Error
            \newline Output Incorrect Error\\ 
        \hline
            \begin{center}
                \textbf{JUMP} 
            \end{center}& 
            \textbf{Summary:} 
            \newline This type of jump is unconditional. When your automata reads this instruction, it will move its instruction 
            counter to the corresponding anchor, effectively skipping forward or backward in your program.

            \textbf{Notes:} 
            \newline This instruction can be used to create loops.

            \textbf{Faults:}
            \newline Output Empty Error
            \newline Output Incorrect Error\\ 
        \hline
    \end{tabular}

    \begin{tabular}{ | m{3cm} | m{11cm} | } 
        \hline
            \begin{center}
                \textbf{JUMP IF NULL} 
            \end{center}& 
            \textbf{Summary:} 
            \newline This type of jump is conditional. When your automata reads this instruction, it will move its instruction counter 
            to the corresponding anchor, if and only if its hands are empty.

            \textbf{Notes:} 
            \newline This instruction can be used to exit loops when there is no more input.

            \textbf{Faults:}
            \newline None\\
        \hline
            \begin{center}
                \textbf{JUMP IF LESS X} 
            \end{center}& 
            \textbf{Summary:} 
            \newline This type of jump is conditional. When your automata reads this instruction, it will move its 
            instruction counter to the corresponding anchor, if and only if the data in its hand is less than the data in register X.

            \textbf{Notes:} 
            \newline This instruction can be used to exit loops or select branches of logic.

            \textbf{Faults:}
            \newline Hand Empty Error
            \newline Register Empty Error\\
        \hline
            \begin{center}
                \textbf{JUMP IF GREATER X} 
            \end{center}& 
            \textbf{Summary:} 
            \newline This type of jump is conditional. When your automata reads this instruction, it will move its instruction counter 
            to the corresponding anchor, if and only if the data in its hand is less than the data in register X.

            \textbf{Notes:} 
            \newline This instruction can be used to exit loops or select branches of logic.

            \textbf{Faults:}
            \newline Hand Empty Error
            \newline Register Empty Error\\
        \hline
    \end{tabular}

    \begin{tabular}{ | m{3cm} | m{11cm} | } 
        \hline
            \begin{center}
                \textbf{MOVETO X} 
            \end{center}& 
            \textbf{Summary:} 
            \newline This instruction allows your automata to place the item in its hand into a register on the board, specified by the argument X.

    
            \textbf{Notes:} 
            \newline This instruction overwrites any data in the target register, and empties your automata’s hand.
    
            \textbf{Faults:}
            \newline Hand Empty Error\\
        \hline
            \begin{center}
                \textbf{COPYTO X} 
            \end{center}& 
            \textbf{Summary:} 
            \newline This instruction allows your automata to copy the item in its hand into a register on the board, specified by the argument X.
    
            \textbf{Notes:} 
            \newline This instruction overwrites any data in the target register, but your automata retains a copy of the data.
    
            \textbf{Faults:}
            \newline Hand Empty Error\\
        \hline
            \begin{center}
                \textbf{MOVEFROM X} 
            \end{center}& 
            \textbf{Summary:} 
            \newline This instruction allows your automata to take an item from a register and keep it in its hand.

    
            \textbf{Notes:} 
            \newline This instruction overwrites any data the automata is holding and removes the data from the target register.
    
            \textbf{Faults:}
            \newline None\\
        \hline
            \begin{center}
                \textbf{COPYFROM X} 
            \end{center}& 
            \textbf{Summary:} 
            \newline This instruction allows your automata to copy an item from a register and keep it in its hand.
    
            \textbf{Notes:} 
            \newline This instruction overwrites any data the automata is holding, but the data also remains in the target register.
    
            \textbf{Faults:}
            \newline None\\
        \hline
    \end{tabular}

    \begin{tabular}{ | m{3cm} | m{11cm} | } 
        \hline
            \begin{center}
                \textbf{ADD X} 
            \end{center}& 
            \textbf{Summary:} 
            \newline This instruction allows your automata to add the value stored in a register to the value stored in its hand.
    
            \textbf{Notes:} 
            \newline The result of the operation is stored back in the automata’s hand.
    
            \textbf{Faults:}
            \newline Register Empty Error\\
        \hline
            \begin{center}
                \textbf{SUBTRACT X} 
            \end{center}& 
            \textbf{Summary:} 
            \newline This instruction allows your automata to subtract the value stored in a register to the value stored in its hand.

            \textbf{Notes:} 
            \newline The result of the operation is stored back in the automata’s hand.

            \textbf{Faults:}
            \newline Register Empty Error\\
        \hline
    \end{tabular}
\end{center}
\newpage