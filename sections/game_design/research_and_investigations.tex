\subsubsection{Papers}
During the prototyping phase of our project, we took the time to study research
centered around teaching computational thinking to beginners, with a special
interest in articles focused on game-based delivery of concepts. We wanted to
know how previous attempts at this approach succeeded and to ascertain which
parts were not successful, what caused them to fail, and how we can avoid the
same pitfalls in our own project. That way, we could make an informed decision
on the progression and clarity of instruction within our game.\\

One study was centered on a program designed to observe students’ abilities in
using Scratch to develop solutions for solving computationally based problems
[1]. Scratch is a drag-and-drop block-style programming interface designed for
children to help them learn how to code without requiring them to navigate the
intricacies of typing and compiling code. The program consisted of a discussion
of the topic coupled with a demonstration in Scratch and an emphasis on the
applicable computational thinking skills required, followed by students creating
their own Scratch based solutions [1]. The programs that the students developed
were evaluated for their ability to decompose the problem at hand and the skill
in efficient program development [1]. The study found that basic concepts like
sequence were easy for students to pick up on, but that as program requirements
became more sophisticated it was increasingly difficult for the participants to
compose and debug their programs [1]. Applying these findings to our own
sequence of instruction has informed us on the importance of properly conveying
the progression of topics. It is essential that we not only deliver clear
information about the mechanics of each element of our game, but also that we
properly articulate and demonstrate the more complex topics, such as nested if
statements or nested loops. The sequence of instructions and data structures
being introduced needs to support a logical progression from simple to more
complex topics, and it is necessary to ensure that players are familiar with
concepts before a new one is introduced. For example, we wouldn’t want to
introduce new instructions in succession, but rather require the player to solve
several puzzles with each element to build familiarity and comfort before a new
element is introduced.  Additionally, we recognize that proper conveyance of
debugging practices and insightful feedback on this practice is an integral
aspect of user success.\\

\todo{(if you have any research on computational thinking that you want to include,
but that isn’t focused specifically on using games to teach it, it should go
before the following paragraph. Make sure you update the references as required
- the list of references must be presented in the same order as each citation
first appears in the documentation, and if any citation moves down on the list,
it’s citation needs to be updated within the documentation)}\\

Other research we found concentrated more specifically on the effectiveness of
using games as a medium for teaching computational thinking to users.
Documentation concentrating on this topic was of special interest to us, as it
applies so directly to our goals for this project. There are already many
studies based around the idea of using game-based instruction for teaching, and
it has been firmly established that it is a successful medium for delivering
instructional content. The real issue we are facing is whether or not it can be
used effectively in regard to teaching computational thinking.\\

One such study we found aimed to teach computational thinking concepts using
“unplugged” games to teach the elements, and then following the unplugged
activities the students would apply what they had learned to correlated
“plugged-in” programming exercises [2]. The unplugged version included tangible
real world objects (e.g., a deck of cards) that students would physically
interact with in order to demonstrate understanding of a particular concept, and
the plugged-in version would present the participants with the same object
computationally and ask them to trace through a solution based on the concept
and rules they were already familiar with for that particular object. The
reasoning for this abstract approach is that the underlying computational
concepts can be learned in any medium, and that by removing the technological
applications the emphasis is specifically on the processing of information using
computational thinking instead of having the participants focus too much on the
technology they are using [2]. It was found that for most of the activities, a
majority of the students were able to accurately solve the tasks on the plugged-
in assessments with approximately 90 percent correctness [2]. However, the
activity that aimed to teach students about conditionals and nested conditionals
had very disappointing results, with students only solving 6 percent of the final
assessment for that lesson correctly [2]. There was also a questionnaire element
to this study designed to help the researchers measure the interest the students
had in pursuing an IT-related job, which the students filled out both before the
course and after it ended. Interestingly, the results actually indicated a
slight decline in students interested in choosing a related job after completing
the course, but contrastingly showed an increase in the desire to learn more
about computer science, and also indicated that students enjoyed the course and
felt they learned something valuable [2]. The shortcomings of this study with
respect to teaching conditionals can likely be attributed to not spending enough
time on the topic and the ending assessment being too complex [2]. This has
greatly informed us on how we want to introduce this topic within our own
project. First, we would like to introduce single conditionals in a tutorial
level and then have the user solve a non-trivial puzzle using the concept,
followed with another tutorial level showing them how to operate with nested
conditionals and a subsequent puzzle implementing that technique. Layering this
instruction instead of releasing it to them all at once should help users
understand the basics of conditionals before allowing them to get bogged down in
more complex applications. It should also be noted that the instructors for this
study were trained in teaching for only two hours a week over the course of
three months [2], whereas, cumulatively, the members of our team have years of
experience in teaching these concepts to novices. Still, the results of the
overall effectiveness of gamification of computational thinking concepts as a
solid method for instruction is very promising, and with the right approach we
know that our project can be successful.\\

\todo{(any research papers you need to report on that apply specifically to
game-based delivery of instruction should follow this paragraph; again, please
make sure you update the references accordingly -- both within this section and
in the references section, by order of appearance)}
