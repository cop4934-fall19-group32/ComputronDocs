\subsection{Goals and Objectives}
\subsubsection{Statement of Intent}
The purpose of this project is to provide a bridge for potential computer
scientists to learn and explore the fundamentals of solving problems
computationally. Providing this insight is non-trivial, especially in
traditional teaching environments. Our approach is to present common
introductory computer science concepts in the form of puzzles and visuals in a
2D video game. This game will provide a means for those interested in
computational thinking to learn in a fun and engaging environment. The players
won’t be expected to set up a development environment, or learn the nuances of
specific programming languages. All of the concepts necessary to solving the
challenges should be taught to (or discovered by) the player through interactive
gameplay. Framing this tool as a game is an important distinction, as it allows
us to deliberately craft the players’ experience and deliver specific learning
outcomes.\\

Currently, there are many game offerings that focus on teaching computational
thinking, with most of them focusing on using “code” solutions for puzzle
solving. However, we have found that they tend to fall far on either end of a
spectrum; either the game is too abstract and the computational thinking aspect
is not well emphasized, or the game is too technical and therefore inaccessible
to beginner level programmers. It is our intention to provide a game that falls
between these two extremes on the spectrum. We want our game to teach the user
about the different commands and how they can be used to solve the puzzles
provided. This will lean away from some of the more abstract puzzle-solvers,
where the movements manipulate a character spatially instead of using the
sequence of commands to solve the puzzle. Conversely, we will use clear and
simple language coupled with visual and audio cues to explain commands and data
structures in the game, while abstracting the intricacies of language specific
commands or complex details about these elements. This will help new programmers
to more easily understand the concepts of computer science and use them to solve
the puzzles without requiring them to be familiar with language specific syntax
or the complexities of managing data structures in code. For example, we can
provide a heap data structure for them to use and explain its general purpose
for sorting and storing data without forcing the user to construct or maintain
the heap. Taking away the technical aspects of utilizing these elements will
allow players to focus on learning and understanding the tools they are given
and how to apply those tools to solve the problem presented.
