\subsection{Ideas}
\subsubsection{Brandy King}
\begin{itemize}
  \item Smaller puzzles can be the parts of a larger cohesive piece
  \begin{itemize}
    \item The efficiency of a solution can dictate how well each piece works
    together
    \item Will reinforce revisiting puzzles that have already been solved to
    find more efficient solutions (leading to a better solution overall)
  \end{itemize}
  \item Variable and Data Structure Declaring
  \begin{itemize}
    \item Separate declaration field from the solution space
    \item Can add or remove variables/structures dynamically
  \end{itemize}
  \item Tutorial levels
  \begin{itemize}
    \item Explaining new instructions or data structures in a non-puzzle level
    \item Allow users to “free play” with their available commands and see how
    things work without the pressure of needing to use them to solve a puzzle
    \item A “to-do” list with checkboxes of key concepts they need for each
    command
    \begin{itemize}
      \item E.G. A Heap tutorial level “to-do” list would include:
      \begin{itemize}
        \item Declare a Heap
        \item Put five inputs in the Heap
        \item Remove five inputs from the Heap
      \end{itemize}
    \end{itemize}
    \item The “to-do” list must be completed before the user can progress to the
    next puzzle, where they will need to utilize the new element in the solution
  \end{itemize}
  \item Points-based scoring of solutions
  \begin{itemize}
    \item Quantify the solution provided with a ranking
    \item Informs users on the efficiency of their solution
  \end{itemize}
  \item Limiting certain commands
  \begin{itemize}
    \item Can’t select to use a data structure space that isn’t declared
    \item Certain commands will not be applicable to data structures
    \begin{itemize}
      \item I.e., data stored in a structure must be removed to be manipulated;
      cannot arbitrarily add or subtract values stored there
    \end{itemize}
    \item Commands with changeable parts will be limited to available locations
  \end{itemize}
  \item Indentation within if statements to emphasize scope
  \begin{itemize}
    \item Will more clearly define to users which commands are reached within
    conditionals; especially important when nested conditionals are introduced
  \end{itemize}
  \item Instruction “Jumbotron”
  \begin{itemize}
    \item An area above the puzzle space that broadcasts which command is being
    executed at that time
  \end{itemize}
\end{itemize}

\subsubsection{John Billingham}
\begin{itemize}
  \item Abstracting conditional jumps with more conventional IF/ELSE statements
  \item Counter that shows how many jumps have been used
  \item Jump iterations used as a finite resource to enforce efficiency
  \item Data Structures as Input/Output
  \begin{itemize}
    \item Arrays, etc
    \item Different data structures unlocked as the puzzle complexity increases
    \item An instruction set associated with each data structure, abstracting
    some of the earlier instructions a bit
  \end{itemize}
  \item Atomic move instructions
  \begin{itemize}
    \item Replacing need to pick up and place down data by combing those into
    one instruction
  \end{itemize}
  \item Add/Subtract specificities
  \item Color-based indexing system
  \item Return instruction
  \item Debugging levels
  \begin{itemize}
    \item Broken solution given initially
    \item Player must fix the broken solution
    \item Can be used in conjunction with tutorials to teach new concepts
  \end{itemize}
\end{itemize}

\subsubsection{Nicolas LaCognata}
\begin{itemize}
  \item Game in the vein of Human Resource Machine
  \begin{itemize}
    \item The player is presented with a limited, assembly-like instruction set
    that they must use to solve logic puzzles.
  \end{itemize}
  \item Data structures as tools
  \begin{itemize}
    \item Data structures are consistently ignored by games in our problem
    space. It would behoove us to take advantage of that gap, and come up with a
    fun and natural way of incorporating data structures into our puzzle format.
  \end{itemize}
  \item Card Game Mechanics
  \begin{itemize}
    \item “Card Playing”
    \begin{itemize}
      \item During our prototyping sessions, I came up with the idea of “Memory
      Cards”. These memory cards would allow the player to declare register
      locations and data structures before their program starts. Making
      mechanics of placing these memory cards similar to a game like Hearthstone
      would be a natural and compelling addition to our puzzle format. This
      “Memory Card” interface would be separate from the normal instruction
      writing interface, which helps reinforce the idea that they are separate
      parts of the puzzle.
    \end{itemize}
    \item “Card packs”
    \begin{itemize}
      \item Card games deliver goodies to players in little packages whose
      contents appear hidden. While discussing adapting card game mechanics, we
      realized that having that little unboxing moment when introducing new
      instructions to the player could be highly compelling.
      \item Other games in our problem space silently drop new instructions on
      the player as they progress. By making the new instructions appear as a
      reward, we can ensure that their addition gets the appropriate attention
      from the player.
    \end{itemize}
  \end{itemize}
\end{itemize}

\subsubsection{Sean Simonian}
\begin{itemize}
  \item Static commands and restricted dynamic commands
  \begin{itemize}
    \item Certain commands in the language should be static, such as “Read
    Input”
    \item Certain commands in the language should have components that can be
    selected from drop down menus, such as “Jump if (blank) (blank) (blank)”
    \begin{itemize}
      \item This command would have 3 sections to specify, so the user could
      specify this command to be “Jump if x > 0”
    \end{itemize}
  \end{itemize}
  \item Break more complex challenges into multiple steps that get checked along
  the way.
  \begin{itemize}
    \item User would have instructions to complete one step of the complete
    level challenge at a time, run their solution, and if they pass then they
    move onto the next step, etc., until they work up to the solution.
    \item This would help players grasp more complex concepts that would be
    harder to teach as a single challenge that they have to break down on their
    own.
    \item Example: Merge Sort algorithm
    \begin{itemize}
      \item Step 1: split an array in half
      \item Step 2: split an array in half recursively
      \item Step 3: compare 2 of the sub arrays and swap them if necessary
      \item Step 4: recursively compare and swap sub arrays to build back up to
      the full sorted array
    \end{itemize}
  \end{itemize}
  \item For testing the educational efficacy of the game, we could create a set
  of short questions to be done on paper that require computational thinking.
  Before a test subject plays our game, we have them work through a few of the
  questions. Then after the test subject plays our game, we have them work
  through a few more questions from the set. Compare the results to see if they
  indicate whether our game helps users with computational thinking and problem
  solving.
  \item For testing the educational efficacy of the game, conducting official
  tests through the UCF psych department would be very beneficial if we are able
  to do so.
  \begin{itemize}
    \item Students taking courses such as Intro to Psychology are required to
    spend a certain number of hours as a test subject as part of their course
    grade. They use SONA to view available tests and sign up for a time slot,
    and the experiments that involve playing video games tend to fill up very
    quickly.
    \item Since Intro to Psychology is a general education course, many subjects
    are freshman, and the majority are not studying computer science or similar
    majors. This could provide an abundant set of candidates that match our
    target audience.
    \item We have to figure out who to contact if we want to set this up for
    next semester.
    \item The UCF psychology department may only allow these official tests to
    be set up and run by psych majors and/or psych graduate students doing
    research. If this is the case, we could look for a grad student conducting
    educational research who would be interested in “sponsoring” these official
    tests.
  \end{itemize}
  \item Incorporate basic cyber security and cryptography concepts into additional
  challenges/puzzles.
  \begin{itemize}
    \item This was a goal early on, but the current design for the game
    mechanics would make this difficult to add to the game.
  \end{itemize}
  \item Initial ideas for computer science concepts to teach. As we narrowed
  down the specifications for our game, these ideas were thrown out as they will
  not fit in the game at all
  \begin{itemize}
    \item Artificial intelligence, neural networks, machine learning
    \item Blockchain technology and its applications
  \end{itemize}
\end{itemize}

\subsection{Papers}
During the prototyping phase of our project, we took the time to study research
centered around teaching computational thinking to beginners, with a special
interest in articles focused on game-based delivery of concepts. We wanted to
know how previous attempts at this approach succeeded and to ascertain which
parts were not successful, what caused them to fail, and how we can avoid the
same pitfalls in our own project. That way, we could make an informed decision
on the progression and clarity of instruction within our game.\\

One study was centered on a program designed to observe students’ abilities in
using Scratch to develop solutions for solving computationally based problems
[1]. Scratch is a drag-and-drop block-style programming interface designed for
children to help them learn how to code without requiring them to navigate the
intricacies of typing and compiling code. The program consisted of a discussion
of the topic coupled with a demonstration in Scratch and an emphasis on the
applicable computational thinking skills required, followed by students creating
their own Scratch based solutions [1]. The programs that the students developed
were evaluated for their ability to decompose the problem at hand and the skill
in efficient program development [1]. The study found that basic concepts like
sequence were easy for students to pick up on, but that as program requirements
became more sophisticated it was increasingly difficult for the participants to
compose and debug their programs [1]. Applying these findings to our own
sequence of instruction has informed us on the importance of properly conveying
the progression of topics. It is essential that we not only deliver clear
information about the mechanics of each element of our game, but also that we
properly articulate and demonstrate the more complex topics, such as nested if
statements or nested loops. The sequence of instructions and data structures
being introduced needs to support a logical progression from simple to more
complex topics, and it is necessary to ensure that players are familiar with
concepts before a new one is introduced. For example, we wouldn’t want to
introduce new instructions in succession, but rather require the player to solve
several puzzles with each element to build familiarity and comfort before a new
element is introduced.  Additionally, we recognize that proper conveyance of
debugging practices and insightful feedback on this practice is an integral
aspect of user success.\\

\todo{(if you have any research on computational thinking that you want to include,
but that isn’t focused specifically on using games to teach it, it should go
before the following paragraph. Make sure you update the references as required
- the list of references must be presented in the same order as each citation
first appears in the documentation, and if any citation moves down on the list,
it’s citation needs to be updated within the documentation)}\\

Other research we found concentrated more specifically on the effectiveness of
using games as a medium for teaching computational thinking to users.
Documentation concentrating on this topic was of special interest to us, as it
applies so directly to our goals for this project. There are already many
studies based around the idea of using game-based instruction for teaching, and
it has been firmly established that it is a successful medium for delivering
instructional content. The real issue we are facing is whether or not it can be
used effectively in regard to teaching computational thinking.\\

One such study we found aimed to teach computational thinking concepts using
“unplugged” games to teach the elements, and then following the unplugged
activities the students would apply what they had learned to correlated
“plugged-in” programming exercises [2]. The unplugged version included tangible
real world objects (e.g., a deck of cards) that students would physically
interact with in order to demonstrate understanding of a particular concept, and
the plugged-in version would present the participants with the same object
computationally and ask them to trace through a solution based on the concept
and rules they were already familiar with for that particular object. The
reasoning for this abstract approach is that the underlying computational
concepts can be learned in any medium, and that by removing the technological
applications the emphasis is specifically on the processing of information using
computational thinking instead of having the participants focus too much on the
technology they are using [2]. It was found that for most of the activities, a
majority of the students were able to accurately solve the tasks on the plugged-
in assessments with approximately 90 percent correctness [2]. However, the
activity that aimed to teach students about conditionals and nested conditionals
had very disappointing results, with students only solving 6 percent of the final
assessment for that lesson correctly [2]. There was also a questionnaire element
to this study designed to help the researchers measure the interest the students
had in pursuing an IT-related job, which the students filled out both before the
course and after it ended. Interestingly, the results actually indicated a
slight decline in students interested in choosing a related job after completing
the course, but contrastingly showed an increase in the desire to learn more
about computer science, and also indicated that students enjoyed the course and
felt they learned something valuable [2]. The shortcomings of this study with
respect to teaching conditionals can likely be attributed to not spending enough
time on the topic and the ending assessment being too complex [2]. This has
greatly informed us on how we want to introduce this topic within our own
project. First, we would like to introduce single conditionals in a tutorial
level and then have the user solve a non-trivial puzzle using the concept,
followed with another tutorial level showing them how to operate with nested
conditionals and a subsequent puzzle implementing that technique. Layering this
instruction instead of releasing it to them all at once should help users
understand the basics of conditionals before allowing them to get bogged down in
more complex applications. It should also be noted that the instructors for this
study were trained in teaching for only two hours a week over the course of
three months [2], whereas, cumulatively, the members of our team have years of
experience in teaching these concepts to novices. Still, the results of the
overall effectiveness of gamification of computational thinking concepts as a
solid method for instruction is very promising, and with the right approach we
know that our project can be successful.\\

\todo{(any research papers you need to report on that apply specifically to
game-based delivery of instruction should follow this paragraph; again, please
make sure you update the references accordingly -- both within this section and
in the references section, by order of appearance)}

\subsection{Initial Game Structure}
\newpage

\subsection{Prototypes}
\subsubsection{Brandy King}
My prototype is called \textit{Desktop} (see Figure \ref{fig:desktop}). Expanding on the initial game structure we formed 
our ideas around, my prototype adds the ability for users to declare variables and data structures. The idea 
is that by making the user responsible for managing these resources directly it will help reinforce the concepts 
associated with each element. An Actor will move around the field and carry out the  commands selected by 
the user in the Solution area.\\

\begin{figure}[!hb]
	\centering
	\includegraphics{desktop}
	\caption{Sample layout of the basic \textit{Desktop} level design}
	\label{fig:desktop}
\end{figure}

\textbf{\textit{Setting}}\\
The playing field of \textit{Desktop} takes place on top of a desk. A small figurine on the desk comes to life at the start 
of the game and moves around the desktop during gameplay. Each level of the game is a puzzle designed to 
test the user on their skills with computational thinking. Tutorial style levels, which occur whenever a new 
command or data structure is introduced, will feature a short to-do list for the player to complete to ensure they
 understand the concept before they are required to use it to solve a puzzle.\\

\textbf{\textit{Elements}}\\
\textbf{Input:}
The input trough will feature a queue of pieces of data that the user will need to manipulate in order to solve 
the puzzle. Data can only be picked up one at a time, and only the topmost piece of data can be selected - the 
player cannot pick up an arbitrary piece from the queue. The values of the data will be randomly generated 
when each level is loaded, and puzzles must be dynamic to handle any number of data elements with any value. 
Data cannot be placed back into the queue once it has been removed. The style of the input trough is to be 
determined.\\

\textbf{Output:}
The output trough is the area where the manipulated pieces of data should be placed in the expected order 
for a correct solution. As an element is added to this queue, it will push the rest of the elements down so that 
the first element is always on the bottom and the most recently added element is always on top. Data cannot 
be retrieved from the queue once it has been placed. The style of the output trough is to be determined.\\

\textbf{Declarations:}
This is the designated area for players to place their declared variables and data structures. These items cannot 
be used in the commands within the solution unless they have been placed here first. The variables and data 
structures will be stored in the commands bank, and players will drag them to the declarations area in order to 
declare them. When the items are dropped here they will snap into the grid. There is a limited amount of space 
available for the player to utilize, so more intricate puzzles will require the player to manage their declarations 
more closely. The declarations area will be styled after a notebook.\\

\textbf{Commands:}
This is a storage area. It will consist of a listed bank of commands, as well as the variable and data structure 
elements, currently available to the user. The commands will appear as small rectangular cards and will be 
manipulated with drag-and-drop controls. The variable and data structure elements will be square shaped to 
make them easily differentiable from the commands. For more information on each of the structures and 
commands, see their respective sections below. The commands area will be styled after a stacked document 
sorter, with loose plans for the player to select which “level” of the sorter they are pulling from - commands 
or data structures. Further separating these two components will help keep the concepts discreet from one 
another.\\

\textbf{Solution:}
This is the area where the player will drag-and-drop their commands in order to build a solution to the puzzle. 
If the solution length exceeds the size of the box, a scroll bar will pop up so that users can navigate through 
their solution. The bottom of this box will have a fixed panel with debugging buttons and a button to run the 
current solution. While a solution is running, an indicator will pop up in the solution box and point at which 
command is currently being carried out by the figurine. The debugging buttons will feature a step button that 
executes only the next line of code, a back button that undoes the last line of code executed, and a stop button 
that will halt the current running solution. Players will be able to rearrange commands within the solution box by 
dragging them to different places within the solution. Similar to how structures will snap to a grid within the 
declarations box, commands will align within the solution box in sequence. The alignment will be dynamic 
whenever a command is being drug through the current sequence of commands so that users can see precisely 
where any inserted commands will be placed. Commands that are inside of an if block will be indented to more 
clearly show the user what is part of the conditional logic. When if blocks are open, the subsequent commands 
will automatically be aligned with the indentation until the user places a command flushed left. This alignment will 
also be dynamic, where users can drag commands right or left to either include them in the block or mark the 
end of the block. The command area will be styled after a yellow legal pad.\\

\textbf{Figurine:}
The figurine will be a small animated character that moves around the desktop, carrying out the commands that 
the user has selected in the solution box. The user will only be able to manipulate the data in the puzzle through 
the actions of the figurine, which are subsequently controlled by the sequence of commands in the solution box. 
The animations of the figurine will assist in conveying how the data is being manipulated by the commands, as 
well as clearly displaying how the sequence of commands in the solution is working to solve the problem. The 
figurine will become visibly upset when a solution fails to find the correct solution, and will cease their actions 
whenever a program halting error is encountered. They will also be the source of providing information to the 
player. If the player clicks on the figurine while they are in a halted state, the figurine can provide feedback to 
the player. When errors halt the figurine, it can inform players on the type of error. If the figurine is at a natural 
halt (ie, the solution finished running but didn’t find the correct solution), it can provide tips and hints to the 
player that can help clarify the goals of the current puzzle. Additionally, this character will be the method of 
delivering newly acquired commands and structures to the player. The exact design of the figurine is to be 
determined.\\

\textbf{Jumbotron:}
The purpose of the Jumbotron is to provide an additional level of clarity to the user. Its sole purpose is to 
essentially broadcast which command the figurine is currently executing. This helps address the issue with 
players only paying attention to the character when running their solutions and not watching the sequence 
being followed within the solution box. The Jumbotron will be styled after a desktop digital clock displaying 
the current command instead of the time, and will flash to indicate a change of command.\\

\textbf{To-Do List:}
The to-do list is located in the upper left hand corner of the desk, and will only become active when the user 
is in a tutorial level. Its purpose is to list the specific tasks that the player needs to accomplish with their new 
command or data structure to demonstrate their understanding of it. Check boxes on the to-do list will automatically 
get ticked off once the task is completed, and players will not be allowed to advance to the next level until all 
tasks on the list are satisfied. During this time, the play area becomes an open play arena, with endless input for 
the player to manipulate. Solutions can be run as normal, but there is no expected correct solution for these levels. 
The player is allowed to freely manipulate data however they want in order to test out their new tools, as well as play 
around with all of the other commands and structures currently available to them. Once all of the items on the to-do 
list are achieved, a button next to the to-do list illuminates and they can choose to move forward to the next puzzle 
whenever they are ready to do so. The to-do list will be styled after a post-it note.\\

\textbf{\textit{Structures}}\\
For all of the structures available to the user, they are not required to maintain the storage of data within the 
structure. Everything that is related to structure and storage will be handled on the back end. This allows the 
user to concentrate solely on the concept of the structure.\\

\textbf{Variable:}
The variable structure can hold a single piece of data. If new data is stored in the variable, the previous value 
being held is lost. Data can be copied from or removed from the variable. The variable will be visualized as a 
single square.\\

\textbf{Heap:}
The player selects if they want a MinHeap or MaxHeap when declaring the heap structure. Multiple pieces of 
data can be stored in a heap. The player is only able to copy or remove the top element from the heap, either 
the min or max depending on which type of heap they are using. When new elements are added to the heap, 
it shakes on screen as it sorts the data. The heap will be visualized as a stack of squares, but only the top 
element will be visible.\\

\textbf{Stack:}
Multiple pieces of data can be stored in a stack. The player can only remove the topmost element from the 
stack. The stack will be visualized as a stack of boxes, but only the top element will be visible. When new 
elements are added to the stack, a new box appears on the top with the most recently added element. Once 
the number of elements in the stack is five additional boxes that are added don’t cause the stack to visually 
expand, but the data will be maintained for any number of elements in the stack.\\

\textbf{Queue:}
Multiple pieces of data can be stored in a queue. The player can only remove the first element from the queue. 
The queue will be visualized as a stack of boxes, but only the first element will be visible. When new elements 
are added to the stack, a new box is slid underneath the current stack, and only the first element of the queue 
is visible. Once the number of elements in the queue is five, additional boxes that are added don’t cause the 
queue to visually expand, but the data will be maintained for any number of elements in the queue.\\

\textbf{\textit{Commands}}\\
Many commands within Desktop have blank fields that the user needs to fill in. Whenever one of these commands
 is first played in the solution box, the available items on the board that can be used to fill that blank are outlined 
with a bright glow, and the player must click on the item they want to use in that blank. The user can also change 
the designated item by clicking on the command itself within the solution box, which will cause the items to illuminate 
again so the user can change the selected item. The applicable items are input and output by default but can also 
include variables and data structures that have been declared. The piece of data that will be manipulated is described 
within the relevant section for that item. If a command is not able to be used on one of these items, it will be explicitly 
stated for that command.\\

\textbf{Pick Up (blank):}
This command will cause the figurine to pick up the applicable data element from the specified item. Any data currently 
being held by the figurine is lost. This command cannot be used on output.\\

\textbf{Put Down (blank):}
This command will cause the figurine to put down the data they are currently holding in the specified item. The figurines 
hands will be empty after this executes. This command cannot be used on input.\\

\textbf{Copy To (blank):}
This command will cause the figurine to make a copy of the data they are currently holding and place the copy in the 
specified item. The figurine will still be holding the original data after this executes. This command cannot be used on 
input or output.\\

\textbf{Copy From (blank):}
This command will cause the figurine to make a copy of the applicable data element from the specified item and hold 
the copy in their hands. Any data held by the figurine before this executes is lost. There are no errors associated with 
this command. This command cannot be used on input or output.\\

\textbf{Jump:}
This command causes the sequence of execution of commands to jump to the specified command. When it is played, 
the user has to drag an arrow attached to the jump command card and hook it into another instruction in the command 
box. There are no errors associated with this command.\\

\textbf{Add:}
This command will have the figurine pick up the applicable data element from the selected item and add it to the data they 
are currently holding. If this command executes when the figurines’ hands are empty, it causes an error. This command 
can only be used on variables.\\

\textbf{Subtract:}
This command will have the figurine pick up the applicable data element from the selected item and subtract it from the 
data they are currently holding. If this command executes when the figurines’ hands are empty, it causes an error. This 
command can only be used on variables.\\

Using if commands in Desktop causes subsequent commands in the solution box to be indented one level to show 
they are part of the conditional block. Users have to drag commands to the left to end the indentation and the conditional 
block.\\

\textbf{If Less Than (blank):}
This command will have the figurine compare the data in their hands to the applicable data element in the specified item. 
If the value of the data they are holding is less than the targeted data, the commands within the if block will be executed. 
Otherwise, the conditional block will be skipped. If this command executes when the figurines’ hands are empty, it causes 
an error. This command cannot be used on input or output.\\

\textbf{If Greater Than (blank):}
This command will have the figurine compare the data in their hands to the applicable data element in the specified item. If 
the value of the data they are holding is greater than the targeted data, the commands within the if block will be executed. 
Otherwise, the conditional block will be skipped. If this command executes when the figurines’ hands are empty, it causes 
an error. This command cannot be used on input or output.\\

\textbf{If Equal To (blank):}
This command will have the figurine compare the data in their hands to the applicable data element in the specified item. If 
the value of the data they are holding is equal to the targeted data, the commands within the if block will be executed. 
Otherwise, the conditional block will be skipped. If this command executes when the figurines’ hands are empty, it causes 
an error. This command cannot be used on input or output.\\

\textbf{\textit{Needed Improvements}}\\
At the time of prototyping, Desktop does not have a way to check whether or not an empty condition might occur. This 
is problematic because we need some form of verification that a specified item is empty, like the input or a particular data 
structure. This missing piece will help bring together the type of algorithmic design and computational thinking that Desktop 
aims to convey to users.\\
\subsubsection{John Billingham}
\textbf{Level Structure Overview}\\

Each level is based on an input to output format. Given an initial input, the player
will be expected to use a set of given instructions to manipulate the data to
match an expected output for the puzzle level they are currently on. If their
created output matches the expected output for that level, they have passed the level.
Otherwise, they will need to come up with a new set of instructions (maybe just
slightly altered from the previous solution) to create the exact expected output.
Example levels can be found in Figure \ref{fig:Find_Sum} and Figure \ref{fig:Return_Last}.\\

Import components of this prototype include:
\begin{itemize}
  \item Registers
  \item Input
  \item Output
  \item Instructions
\end{itemize}

\textbf{Registers}\\

The puzzle game will rely on a color-based memory indexing system to access and
manipulate game data. Colors will be used when accessing data structure indices,
although this only applies to registers for now. Two basic blue and red registers
can be seen in Figure \ref{fig:Register_Examples}.

\begin{figure}[!hb]
  \caption{Register Examples}
  \label{fig:Register_Examples}
  \centering
  \includegraphics[scale=0.6]{JB_Prototype_Registers.png}
\end{figure}

Consider being shown these two colored boxes. The contents of the box are represented
as the large, centered numbers in bold. The indices of the box are represented as the
small number at the bottom right-hand corner. Indices tell you where to look in a
data structure (redundant for now, since registers only get one index).

\begin{itemize}
  \item Figure \ref{fig:Register_Indexing} shows how to access the 9 in the blue register.
  \begin{itemize}
    \item (called: blue sub 0)
  \end{itemize}
  \item Figure \ref{fig:Register_Indexing} also shows how to access the 2 in the red register.
  \begin{itemize}
    \item (called: red sub 0)
  \end{itemize}
\end{itemize}

\begin{figure}[!hb]
  \caption{Register Indexing}
  \label{fig:Register_Indexing}
  \centering
  \includegraphics[scale=0.6]{JB_Prototype_Register_Indexing.png}
\end{figure}

From now on, these data structures of size 1, that only contain the index 0, will
be referred to as registers. A specific register and its index is said to be a
memory location.\\

\textbf{Input}\\

Input is data that is given to you initially. This may include one register or
multiple registers, as shown in Figure \ref{fig:Input_Example}.\\

\begin{figure}[!hb]
  \caption{Input Examples}
  \label{fig:Input_Example}
  \centering
  \includegraphics[scale=0.9]{JB_Prototype_Input.png}
\end{figure}

Input can also arrive in the form of an input stream. A stream of input is connected
to one register and continually pushes data from the stream into the register as
data is removed from it (see Figure \ref{fig:Input_Stream}). If there is no more data in the stream,
nothing will be pushed to the register.

\begin{figure}[!hb]
  \caption{Input Stream}
  \label{fig:Input_Stream}
  \centering
  \includegraphics[scale=1]{JB_Prototype_Input_Stream.png}
\end{figure}
\vfill
\clearpage

\textbf{Output}\\

The output is checked at the end of the program. It is up to the player to link
the output to the correct data structure (only registers for now), as well as to
make sure the contents of the data structure match the expected output for the level.
A basic register to output linking can be seen in Figure \ref{fig:Output_Basic}.

\begin{figure}[!hb]
  \caption{Register to Output}
  \label{fig:Output_Basic}
  \centering
  \includegraphics[scale=0.6]{JB_Prototype_Output_Basic.png}
\end{figure}

The green circle specifies that the contents of the green data structure will be
returned as the output for the player's solution to the puzzle level.\\

Consider a scenario in which multiple data structures may be returned as output.
In the following example (Figure \ref{fig:Output_Choice}), the player needs to return the register
with the greater value by choosing which colored data structure to link to the output.

\begin{figure}[!hb]
  \caption{Output Choice}
  \label{fig:Output_Choice}
  \centering
  \includegraphics[scale=0.7]{JB_Prototype_Output_Choice.png}
\end{figure}

Players can link data structures to output by simply choosing the corresponding colored
circle of the data structure they would like to return. Clearly 10 is greater than 2 here and
the red register should returned.\\

\textbf{Instructions}\\

The notable instructions are as follows:

\begin{itemize}
  \item Move
  \item Return
  \item Add
  \item Subtract
  \item Jump
\end{itemize}

Instructions are clicked or dragged from a limited set of given instructions, 
specific to each level.\\

When the player is ready to test their solution, each instruction will execute
procedurally from top to bottom.\\

It is up to us to structure the levels and instructions given in each level in such
a way that minimizes the amount of errors the player can create. Runtime errors
are okay, but we should strive to avoid having the player feel as if they are learning
syntax. We would also like to avoid confusing the player by giving a limited
instruction set for each level with thorough explanations for each instruction.\\

Parameters used in certain instructions (Move, Return, Add, Subtract) are chosen
from a drop down menu, again minimizing the amount of errors the player can create
on their own. When choosing a colored data structure and an index, only the colors of
the data structures and their valid indices that are already on the puzzle board
are given as options in the drop down menu.\\
\newpage

\textbf{Instruction: Move}\\

The move instruction takes two arguments, a source register and a destination register.
It's main functionality consists of three different parts:

\begin{itemize}
  \item Move the contents of the source to the destination
  \item Leave the contents of the source unchanged
  \item Override the previous contents of the destination
\end{itemize}

How the instruction would look in the instruction editor is shown in Figure
\ref{fig:Move_Instruction}.

\begin{figure}[!hb]
  \caption{Move Instruction}
  \label{fig:Move_Instruction}
  \centering
  \includegraphics[scale=1]{JB_Prototype_Move.png}
\end{figure}

The contents of the two registers before and after the move instruction is
received is shown in Figure \ref{fig:Move_Instruction_Use}.

\begin{figure}[!hb]
  \caption{Move Instruction In Use}
  \label{fig:Move_Instruction_Use}
  \centering
  \includegraphics[scale=0.8]{JB_Prototype_Move_Use.png}
\end{figure}

Each argument (left and right side of the arrow) will have two drop down menus:
\begin{itemize}
  \item Color of register
  \item Index of register
\end{itemize}

The move instruction is atomic and meant to abstract the process of manually
picking up the data and placing it down.\\

\textbf{Instruction: Return}\\

The return instruction will specify only one argument -- which colored data structure
to return as output.
\begin{itemize}
  \item A return instruction is required for every level
  \item May be given, or may require the player to write it
  \item The data structure that is returned is compared against the expected
  output for the level
\end{itemize}

Figure \ref{fig:Return_Instruction} shows how the return instruction looks in the instruction editor.

\begin{figure}[!hb]
  \caption{Return Instruction}
  \label{fig:Return_Instruction}
  \centering
  \includegraphics[scale=0.8]{JB_Prototype_Return.png}
\end{figure}

The argument is chosen from a drop down menu that allows the player to select
the color of the data structure they want to return.\\

Using the example from the output section (Figure \ref{fig:Output_Choice}),
we can show the correct usage of the return instruction in Figure \ref{fig:Return_Instruction_Use}.

\begin{figure}[!hb]
  \caption{Return Instruction In Use}
  \label{fig:Return_Instruction_Use}
  \centering
  \includegraphics[scale=0.7]{JB_Prototype_Return_Use.png}
\end{figure}
\vfill
\clearpage

\textbf{Instruction: Add}\\

The add instruction takes three arguments:
\begin{itemize}
  \item Two memory locations to be added
  \item The destination of their sum, another memory location
\end{itemize}

Figure \ref{fig:Add_Instruction} shows how the add instruction looks in the
instruction editor.

\begin{figure}[!hb]
  \caption{Add Instruction}
  \label{fig:Add_Instruction}
  \centering
  \includegraphics[scale=0.8]{JB_Prototype_Add.png}
\end{figure}

When referencing the memory locations, an index must also be specified, as we've
seen before. Figure \ref{fig:Add_Instruction_Use} shows the add instruction in use by displaying
the contents of the registers included in the instruction before and after it is
run.

\begin{figure}[!hb]
  \caption{Add Instruction In Use}
  \label{fig:Add_Instruction_Use}
  \centering
  \includegraphics[scale=0.7]{JB_Prototype_Add_Use.png}
\end{figure}

Note:\\
The destination can be one of the source registers used, in which case it would
just overwrite the data, similarly to how the move instruction functions.\\
\newpage

\textbf{Instruction: Subtract}\\

The subtract instruction works exactly the same as the add instruction above,
although the difference of the two registers is placed into the destination, rather
than the sum.\\

Figure \ref{fig:Sub_Instruction} shows how the subtract instruction looks in the
instruction editor.

\begin{figure}[!hb]
  \caption{Subtract Instruction}
  \label{fig:Sub_Instruction}
  \centering
  \includegraphics[scale=0.9]{JB_Prototype_Sub.png}
\end{figure}

\textbf{Instruction: Jump}\\

The jump instruction is paired with an anchor. When a jump instruction is executed,
the flow of control immediately jumps to the instruction that lays at the anchor.\\

It is up to the user to choose where the anchor will lie after placing a Jump instruction
in the instruction editor.\\

Figure \ref{fig:Jump_Instruction} shows how the jump instruction looks in the
instruction editor.

\begin{figure}[!hb]
  \caption{Jump Instruction}
  \label{fig:Jump_Instruction}
  \centering
  \includegraphics[scale=0.9]{JB_Prototype_Jump.png}
\end{figure}

Figure \ref{fig:Jump_Instruction_Use} shows how the control flow is handled when a jump instruction is
executed.

\begin{figure}[!hb]
  \caption{Jump Instruction In Use}
  \label{fig:Jump_Instruction_Use}
  \centering
  \includegraphics[scale=0.9]{JB_Prototype_Jump_Use.png}
\end{figure}
\vfill
\clearpage

\begin{figure}[!hb]
  \caption{Example Level: Find Sum}
  \label{fig:Find_Sum}
  \centering
  \includegraphics[scale=1]{JB_Prototype_Find_Sum.png}
\end{figure}
\vfill
\clearpage

\begin{figure}[!hb]
  \caption{Example Level: Return Last}
  \label{fig:Return_Last}
  \centering
  \includegraphics[scale=1]{JB_Prototype_Return_Last.png}
\end{figure}
\vfill
\clearpage

\newpage

\subsubsection{Nicolas LaCognata}
For my prototype, I spent a lot of time focusing on how we should integrate data structures into our assembly-based
puzzle structure. The approach I took saw players attaching data structures to registers to 'augment" them and
change how the register stores and processes information. I also took a stab at defining a theme for our game.\\

\textbf{Setting}\\
In Automata, you play an Automaton stuck in a laboratory who is tasked with reprogramming itself to solve problems.
When solving the puzzle, they player may be addressed by a narrator when they make a mistake. The game was to have 
a "Mad Science" feel to it, much like \textit{Portal}.\\

\textbf{Rules}\\
For each puzzle the player is presented with a problem, and they must use the provided instructions to solve it. 
Every puzzle will contain the following elements:

\begin{itemize}
    \item An Input track
    \item An Output track
    \item A set of static registers built into the test chamber
    \begin{itemize}
        \item These resisters all the player to store manage multiple data cubes
    \end{itemize}
    \item A set of register augmenters
    \begin{itemize}
        \item These augmenters turn the registers into special data structures that can make the puzzles easier/possible.
    \end{itemize}
    \item An instruction set
    \begin{itemize}
        \item These are the instructions the automata can execute to solve the puzzle. When placed in the solution window,
        they will be executed in the order they appear
    \end{itemize}
    \item Some operations are invalid for certain states of the puzzle. For your safety, please avoid executing invalid operations.
\end{itemize}

\textbf{Instruction Set}
\begin{center}
    \begin{tabular}{ | m{3cm} | m{11cm} | } 
        \hline
            \begin{center}
                \textbf{INPUT} 
            \end{center}& 
            \textbf{Summary:} 
            \newline Allows your automata to grab an item from the input track. It places it in its hands.

            \textbf{Notes:} 
            \newline Your automata will drop any item currently in their hands to accept the new input, losing the old data forever.
            \newline If the input track is empty, the automata will do nothing

            \textbf{Faults:}
            \newline None\\
        \hline
            \begin{center}
                \textbf{OUTPUT} 
            \end{center}& 
            \textbf{Summary:} 
            \newline Allows your automata to place the item in its register into the output track.

            \textbf{Notes:} 
            \newline None

            \textbf{Faults:}
            \newline None\\
        \hline
            \begin{center}
                \textbf{SUBMIT} 
            \end{center}& 
            \textbf{Summary:} 
            \newline Allows your automata to submit its output for evaluation.

            \textbf{Notes:} 
            \newline This should always be the last instruction in your program.

            \textbf{Faults:}
            \newline Output Empty Error
            \newline Output Incorrect Error\\ 
        \hline
            \begin{center}
                \textbf{JUMP} 
            \end{center}& 
            \textbf{Summary:} 
            \newline This type of jump is unconditional. When your automata read this instruction, it will move its instruction 
            counter to the corresponding anchor, effectively skipping forward or backward in your program.

            \textbf{Notes:} 
            \newline This instruction can be used to create loops.

            \textbf{Faults:}
            \newline Output Empty Error
            \newline Output Incorrect Error\\ 
        \hline
    \end{tabular}

    \begin{tabular}{ | m{3cm} | m{11cm} | } 
        \hline
            \begin{center}
                \textbf{JUMP IF NULL} 
            \end{center}& 
            \textbf{Summary:} 
            \newline This type of jump is conditional. When your automata read this instruction, it will move its instruction counter 
            to the corresponding anchor, if (and only if) its hands are empty.

            \textbf{Notes:} 
            \newline This instruction can be used to exit loops when there is no more input.

            \textbf{Faults:}
            \newline None\\
        \hline
            \begin{center}
                \textbf{JUMP IF LESS X} 
            \end{center}& 
            \textbf{Summary:} 
            \newline This type of jump is conditional. When your automata read this instruction, it will move its 
            instruction counter to the corresponding anchor, if (and only if) the data in its hand is less than the data in register X.

            \textbf{Notes:} 
            \newline This instruction can be used to exit loops or select branches of logic.

            \textbf{Faults:}
            \newline Hand Empty Error
            \newline Register Empty Error\\
        \hline
            \begin{center}
                \textbf{JUMP IF GREATER X} 
            \end{center}& 
            \textbf{Summary:} 
            \newline This type of jump is conditional. When your automata read this instruction, it will move its instruction counter 
            to the corresponding anchor, if (and only if) the data in its hand is less than the data in register X.

            \textbf{Notes:} 
            \newline This instruction can be used to exit loops or select branches of logic.

            \textbf{Faults:}
            \newline Hand Empty Error
            \newline Register Empty Error\\
        \hline
    \end{tabular}

    \begin{tabular}{ | m{3cm} | m{11cm} | } 
        \hline
            \begin{center}
                \textbf{MOVETO X} 
            \end{center}& 
            \textbf{Summary:} 
            \newline This instruction allows your automata to place the items in its hand into a register on the board, specified by the argument X.

    
            \textbf{Notes:} 
            \newline This instruction overwrites any data in the target register, and empties your automata’s hand.
    
            \textbf{Faults:}
            \newline Hand Empty Error\\
        \hline
            \begin{center}
                \textbf{COPYTO X} 
            \end{center}& 
            \textbf{Summary:} 
            \newline This instruction allows your automata to copy the items in its hand into a register on the board, specified by the argument X.
    
            \textbf{Notes:} 
            \newline This instruction overwrites any data in the target register, your automata retains a copy of the data.
    
            \textbf{Faults:}
            \newline Hand Empty Error\\
        \hline
            \begin{center}
                \textbf{MOVEFROM X} 
            \end{center}& 
            \textbf{Summary:} 
            \newline This instruction allows your automata to take an item from a register and keep it in its hand.

    
            \textbf{Notes:} 
            \newline This instruction overwrites any data the automata is holding and removes the data from the target register.
    
            \textbf{Faults:}
            \newline None\\
        \hline
            \begin{center}
                \textbf{COPYFROM X} 
            \end{center}& 
            \textbf{Summary:} 
            \newline This instruction allows your automata to copy an item from a register and keep it in its hand.
    
            \textbf{Notes:} 
            \newline This instruction overwrites any data the automata is holding, the data also remains in the target register.
    
            \textbf{Faults:}
            \newline None\\
        \hline
    \end{tabular}

    \begin{tabular}{ | m{3cm} | m{11cm} | } 
        \hline
            \begin{center}
                \textbf{ADD X} 
            \end{center}& 
            \textbf{Summary:} 
            \newline This instruction allows your automata to add the value stored in a register to the value stored in its hand.
    
            \textbf{Notes:} 
            \newline The result of the operation is stored back in the automata’s hand.
    
            \textbf{Faults:}
            \newline Register Empty Error\\
        \hline
            \begin{center}
                \textbf{ADD X} 
            \end{center}& 
            \textbf{Summary:} 
            \newline This instruction allows your automata to subtract the value stored in a register to the value stored in its hand.

            \textbf{Notes:} 
            \newline The result of the operation is stored back in the automata’s hand.

            \textbf{Faults:}
            \newline Register Empty Error\\
        \hline
    \end{tabular}
\end{center}
\newpage
\subsubsection{Sean Simonian}
\textbf{Paper Prototype 1.0}

My initial paper prototype, shown in Figure \ref{fig:Paper_Prototype_1.0}, built on the established game structure of manipulating input and output by executing instructions in a player's solution. The objective for each level is displayed at the top left of the screen, the solution space is on the right side, and the remaining area is utilized for visualizing the command execution. This prototype includes a robotic claw that can move horizontally across the top of the main game UI area, and can extend vertically to pick up or drop input values.

\begin{figure}[!hb]
  \caption{Paper Prototype 1.0}
  \label{fig:Paper_Prototype_1.0}
  \centering
  \includegraphics[scale=0.45]{SS_PP1.png}
\end{figure}


\paragraph{Solution Space:} ~\\
The solution space on the right side of the screen is where the player constructs their code-like solution to satisfy the puzzle objectives.
The player must use this space to place a series of instructions that will execute in order.

\paragraph{Instructions/commands:} ~\\
The instructions that are available in this prototype are:
\begin{itemize}
	\item Read Input
	\begin{itemize}
		\item The claw picks up the topmost item from the input.
	\end{itemize}
	\item Jump [If \_\_\_]
	\begin{itemize}
		\item This instruction will point to another line in the solution to jump to while executing.
		\item Optionally, the player may include an If \textit{condition}, so that the jump will only occur if a condition is met.
	\end{itemize}
	\item Copy \_\_\_ \_\_\_
	\begin{itemize}
		\item Copies the current value in the claw to a specified register, or copies the value from a register to the claw.
		\item The first blank can be set to either ``to'' or ``from''
		\item The second value can be set to a specific register
		\item Example: \textit{Copy to register 0}
	\end{itemize}
	\item Add \_\_\_
	\begin{itemize}
		\item Add the value of a specified register to the current value in the claw.
	\end{itemize}
	\item Subtract \_\_\_
	\begin{itemize}
		\item Subtract the value of a specified register from the current value in the claw.
	\end{itemize}
	\item Return
	\begin{itemize}
		\item Place the current value held by the claw into the return bin, and stop executing the instructions.
	\end{itemize}
\end{itemize}


\paragraph{The Claw:} ~\\
As each step of the player's solution is executed, the claw manipulates the data on the board, and the corresponding instructions are highlighted in the solution space.
The claw does all of the work manipulating data as it simulates the player's solution to the puzzle, and it can only hold at most one value at any time.
By having the claw hold a value and manipulate is as specified in the solution, this renders a separate ``current value'' box unnecessary.
The claw can pick up and put down data elements, or it can combine the element it's currently holding with another element to add or subtract values.

\paragraph{Input:} ~\\
Input arrives in a vertical stack that acts like a conveyor belt. The claw will navigate over the input stack, pick up the topmost item, and proceed to manipulate the data.

\paragraph{Registers:} ~\\
Registers are located in the middle of the game board, and they can be used to store data that needs to be set aside while the claw manipulates other data elements.

\paragraph{Discard Bin:} ~\\
Sometimes the current value being held by the claw is no longer needed, so it can be discarded in a trash can towards the bottom right of the screen. This gets rid of the value and can free up space for the claw or registers to hold useful elements.

\paragraph{Puzzle Description:} ~\\
Every puzzle has a clearly defined objective that should result in a single value to return upon completion, similar to a return value of a function in traditional programming. This prototype includes a ``return'' bin next to the discard bin, and the claw will place the final return value in the return bin to end the code execution. If the return value matches the expected value, then the solution was successful, otherwise the solution is incorrect.\\



\textbf{Paper Prototype 2.0: Puzzle Scene Design Concept}

After multiple iterations of prototype testing, team discussions to determine what works and what should be changed, we shared a solid view of how the puzzles should work, the components of included in the puzzles, the complete instruction set, etc.
This prototype was designed to present a possible view of what the entire level will look like, where the components are placed, how they interact with each other on screen, etc.

\todo{Add diagram of prototype and reference above}



\paragraph{Menu Bar:} ~\\
The menu bar stretches across the top of the screen. This contains the name of the current level, along with some buttons to the right.
The Menu button will pause the game and bring up a pause menu, which will present the player with options to exit the puzzle, change game options, or resume the game.
The Help button can be pressed to give useful hints or tips for solving the puzzle if the player is struggling.


\paragraph{Level Description:} ~\\
Below the top menu bar is a section for the the description of the current level objectives. This will contain text that explains the current puzzle in a concise manner.


\paragraph{Solution Space:} ~\\
Similar to previous prototypes, the solution space on the right side is where the player constructs a solution to solve the puzzle objectives. This space includes a scrollbar on the right side to navigate long solutions.


\paragraph{Instruction Set:} ~\\
Below the solution space is the set of instructions the player can use for the current level.
The player can drag and drop these instructions into the solution space to construct their solution.
Dragging an instruction from the set creates a new instance of that instruction, meaning that the instruction does not get removed from the instruction set.
If the player clicks on an instruction, a description of that instruction will appear on the screen to remind the player what the instruction does and how to use it.


\paragraph{Game Space:} ~\\
The majority of the screen will be the game space, which contains the remaining elements which are explained below. The game space is where the remaining components interact to visually simulate the steps as the player's solution is executed.


\paragraph{The Claw:} ~\\
As each step of the player's solution is executed, the claw moves data between game components.
The claw moves horizontally along the top of the game space, and can expand/retract to reach components below it.
The claw is only used for transportation of data between two points, it does not hold onto anything when it is idle.
This means that any data that was picked up will be put down somewhere else before moving to the next step. Data is not modified while it is in the claw.


\paragraph{Computer:} ~\\
The central figure of this prototype design a sentient computer as a sort of actor.
The computer holds the current value that is displayed on its monitor, and it performs computations and manipulates data.
The claw can deliver data to the computer or pull data from the computer to move to another game component, such as a data structure or output.


\paragraph{Input:} ~\\
Input is received from a vertical stack or conveyor belt system on the bottom left of the game space.
The claw can pick up the topmost input value and move it somewhere else.


\paragraph{Output:} ~\\
Output is delivered to a vertical stack or conveyor belt system on the bottom right of the game space.
The claw can place a value onto the top of the output, and any prior output values slide downward or off the screen.


\paragraph{Card Space:} ~\\
The card space is at the bottom of the game space between input and output, and it displays the data structure cards that are available or in use.
The player can ``play'' a card by dragging and dropping an available card to the ``in use'' section.
Cards being used will display their contents, or if the data structure is too large to display, the player can click on the card to enlarge it and view the contents.
Players can also click on a card to get a description of the data structure and how to use it.



\newpage


