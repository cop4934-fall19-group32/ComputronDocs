\subsubsection{Player Controls}

As the player engages with the game, they will be able to control the simulation of their solution as it is executed in the puzzle game space. Once the player is ready to test out a solution, they will press a button to begin execution. At this point, the interpreter will get the instructions from the UI and determine the sequence of steps that will be simulate the solution. This gets communicated to the actor to determine the moves it will take across the game space. As the actor progresses through its actions, it communicates with the UI to grab data from UI elements and move data to other UI elements. These simulation steps cannot just be discarded once they are completed, since the player has controls to pause the simulation, proceed, or rewind to previous steps. These player controls will require the UI to communicate with the actor, so that the actor knows if it should continue with the next action, stop, or rewind the previous action.






\subsubsection{Solution Pane}

The solution pane is where the player must utilize available instructions to construct a solution to the current puzzle. The set of available instructions is received from the puzzle system to the UI. When the player is ready to execute their solution, the sequence of instruction commands is sent to the interpreter to analyze and decide what actions will be taken.
In order for this to work smoothly, the UI and interpreter need to communicate clearly what as the puzzle simulations execute.







\subsubsection{Player Messaging}

The UI needs to receive a number of messages to display to help the player when certain conditions are met during gameplay. The puzzle system will store most of these messages for each level, and some will go directly from the puzzle system to the UI, such as the level description.
The UI will also get messages from the interpreter, such as an error when trying to "compile" the player's solution, or messages to reflect conditions while executing.
The actor may have messages to communicate to the UI, such as hints for the level, or encouraging messages that should be displayed for the player.
The UI will be communicating with the other major components in order for player messaging to work successfully.



