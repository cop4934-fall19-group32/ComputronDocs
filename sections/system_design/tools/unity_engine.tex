Making games from scratch is a long and difficult practice. Like most small development 
teams, we decided very early on that making a custom framework or engine would be 
inappropriate for the scope of the project. The team opted to instead make use of a 
commercial game engine. Using a commercial game engine puts a huge set of tools at 
our disposal, and allows for rapid prototyping and iterations. However, choosing which 
game engine to use is still a non-trivial issue. In the realm of commercial game engines, 
there are two major options: Unreal Engine 4 (developed by Epic Games) and Unity 
(developed by Unity Technologies). Both engines are excellent choices, and we carefully 
weighed their positive and negative attributes.\\

Unreal Engine 4 is a high power tool. Written and scripted in C++, games shipped with 
Unreal can push the limits of the player's computer hardware. In addition Unreal is open 
source, includes better support for version control, and provides a very mature visual 
scripting system to aid in producing game logic. However, concerns were raised that 
Unreal would be too powerful for the needs of our project. Unreal's higher graphical 
fidelity is irrelevant in a sprite-based 2D application, and the engine's generally steeper 
learning curve threatened to greatly slow our progress in the project's early stages. In 
addition, Unreal's huge C++ code base can often result in very long compile times, 
especially for older development machines.\\

Unity is also an extremely advanced tool. Written in C++ and scripted with C\#, games 
made with Unity can generally be prototyped and iterated much faster than Unreal 
competitors. Unity is easier to learn, has better documentation and more tutorials, and 
has vastly superior support for 2D titles. However, Unity presents its own set of drawbacks 
as well. Poor support for modern version control systems, the lack of a visual scripting tool, 
closed source engine code, and a dated pricing model were all worrisome issues.\\

After extensive discussions, the team unanimously decided that \textit{Computron} will be 
built using Unity, version 2019.2.12.f1. Improved support for 2D titles and and superior 
ease of use were the deciding factors.\\