\subsubsection{Overview}
The Actor will be responsible for simulating the execution of the solution constructed
by the player through movement and animations. It is an animated character in the game that will move around the Puzzle Scene and manipulate the data elements of the current puzzle based on the sequence of commands that the player constructs in the Solution Pane of the User Interface. \\

Figure \ref{fig:actor_diagram} provides a high-level outline of the Actor component and how it 
interacts with the other core aspects of the game.\\

\begin{figure}[!hb]
  \caption{Actor Component Overview}
  \label{fig:actor_diagram}
  \centering
  \includegraphics[scale=0.6]{Diagrams/actor_diagram.png}
\end{figure}

The responsibilities of the Actor are as follows:

\begin{itemize}
	\item Retrieve commands from the Interpreter component
	\item Alert the Interpreter component to halt execution when an invalid command is attempted
	\item Move to the correct waypoints based on the command received
	\item Properly handle data manipulation in the puzzle scene (move and update data elements
			as specified by the current command)
	\item Communicate the flow of execution to the player through animations
	\item Illustrate to the player how data changes in response to commands via animations
	\item Act as a liason to the player for hints and other important messages.
\end{itemize}

\subsubsection{Interpreter Interactions}
Effective integration with the Interpreter is a key component of the functionality of the Actor. The Actor relies completely on the command sequence from the Interpreter in order to illustrate the execution of the solution. The Actor will continue to retrieve the next command in sequence from the Interpreter until the sequence terminates or an invalid command is attempted. If the command can be executed, the Actor will carry out the instruction visually so that the player can understand the flow of execution and the logic behind their solution. If the command causes an error in execution, the Actor will alert the Interpreter to halt execution. The flow of this process between the Actor and Interpreter can be seen in Figure \ref{fig:interpreter_interactions}.\\

\begin{figure}[h]
  \caption{Interactions between the Interpreter and Actor components}
  \label{fig:interpreter_interactions}
  \centering
  \includegraphics[scale=0.6]{Diagrams/interpreter_interactions.png}
\end{figure}

\subsubsection{Processing Commands}
The processing of commands is the core function of the Actor. Interfacing with the Interpreter component is only the first part of command processing; the majority of that processing occurs independently within the Actor component once a command has been retrieved. Figure \ref{fig:processing_commands} shows in detail how the Actor internally processes commands.\\

\begin{figure}[h]
  \caption{The state diagram followed by the actor when processing commands}
  \label{fig:processing_commands}
  \centering
  \includegraphics[scale=0.6]{Diagrams/processing_commands.png}
\end{figure}

After fetching a command, the Actor will then execute a series of movements and actions to carry out that command. This includes moving to the proper waypoint to initiate the instruction, manipulating data from the Puzzle Scene object the Actor is currently at, moving to a second waypoint to continue execution, and manipulating data at that second Puzzle Scene object. Depending on what the current command is and the status of the actor when the new command is retrieved, some of these four steps may not be applicable. Parsing each instruction separately and successfully regardless of the current state is vital. Commands that cause errors must be reported accurately back to the Interpreter so that execution can be halted when they are encountered, and no false reporting can be tolerated. If the command won't cause an error until it is partially completed, the partial execution of that command must be shown. Additionally, commands that can complete without causing errors should always be carried out exactly as specified. This is integral to the process of the game because even if the command is the wrong selection at the time, it still needs to be executed properly in the puzzle to show the player the error in their logic. Of course, correct commands that work towards the solution also need to be executed properly to eventually reach the solved state of the puzzle.\\

\subsubsection{Player Messaging}
The Actor is also the main form of direct communication with the player. Through the visual execution of the instructions that the player has selected, the Actor conveys how the solution the player built functions. This includes not just moving around the board and manipulating the data elements, but also providing visual cues to the user via detailed animations. For example, the data elements will be visibily picked up and held by the Actor. If that data is copied to another location, the Actor will be seen producing a copy of that data and placing it at the location specified. Data that is incremented or decremented via commands will be changed according to those commands so that the player sees and understands how the data elements are being manipulated. Additionally, providing expressive and emotive animations enhances the clarity of communication without bogging down the player with too much text. If the current command is to select the next element from Input but the Input box is empty, the Actor would attempt to pick up the item, find that nothing is there, and then emphasize that their hands are empty.\\

Aside from effective visual communication, the Actor will also provide information to the player via text when necessary. This form of communication will be used sparingly and will always be coupled with visual cues to help cut down on the amount of explicit text required to convey the information to the player. Because the Actor is essentially the only character in the game as well as the focus of the game action, any dynamic messaging that needs to occur will come from the Actor. Static messaging that does not change based on current activity will be left to the User Interface. Expanding on this idea of the Actor being a point of contact, if the player finds they are stuck on a certain puzzle they can click on the Actor in the field and request a hint on the solution for the current puzzle. \\