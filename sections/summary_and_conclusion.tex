\subsection{Summary}
The early stages of our project mainly consisted of research and testing. When we first began, we knew the main idea of our project -- an educational video game that taught players Computer Science concepts -- but we didn't have a clear definition on how that idea would be realized or the steps we needed to take to get there. We knew we didn't want to jump into designing the game and its mechanics without a clearly defined plan and endpoint in mind. As a team we combined efforts in researching and brainstorming to determine a general concept that our game would be based around. With a central idea defined, we each developed a paper prototype mock-up of what our version of the game would be and playtested it on people inexperienced with programming. The unified prototype was borne of our combined efforts and the information gathered from this initial research and development phase of \textit{Computron}.\\

After the unified prototype was formed, we each produced an identical whiteboard version of the game and tested it on as many people as we could. The testing went exceptionally well and we all agreed that we were ready to move forward with coding the game systems and mechanics. With a clear picture now forming as we began laying down the skeleton code to instantiate the basic layout of our game, we continued to fine tune the definitions and expectations of each aspect of our game and what the end result product should look like. This consisted not just of the physical appearance and feel of the game, but also the functionality of each component and how it should interact with other core components of the game. Additionally, the progression of puzzles that was designed for use with the prototype tested so well that it was adopted directly as the tutorial sequence for \textit{Computron}.\\

Moving forward, as we continue to code and develop our game, we will also maintain the momentum we have with playtesting the whiteboard prototypes. So far, the testing has given us invaluable insight into how players want to interact with our game, which aspects of the design work well, which ones need more explanation, and how effective our approach is in teaching the concepts desired. This also helps us test out possible changes to our design before we code them, saving us from unnecessary reworking.\\

\subsection{Conclusion}
This semester has been immensely successful for our team. By doing research and testing upfront to clearly define a vision for our game, we have eliminated the risk of coding ourselves into a corner and needing to start over or completely restructure the game system further into the life of the project. Additionally, the whiteboard version of our game is portable and easily testable with potential users. By running through the whiteboard version of our game with players from a variety of skill levels, we have been able to accumulate feedback that goes directly into informing the design of our game.\\

Our team has met and surpassed all of our project milestones to date, and we have a detailed plan laid out to keep us on track to preserve the progress we are making without compromising the quality or integrity of our project. A fully playable version of our game is due to be completed in March so that we can move the playtesting focus from the whiteboard version to the final computer-based version. The remaining phase of the project will consist mostly of fine-tuning and polishing based on the findings of the computer playtesting sessions.\\