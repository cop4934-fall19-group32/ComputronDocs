\subsection{Summary}
The early stages of our project mainly consisted of research and testing. When we first began, we knew the main idea of our project -- an educational video game that taught players computer science concepts -- but we didn't have a clear definition on how that idea would be realized or the steps we needed to take to get there. We knew we didn't want to jump into designing the game and its mechanics without a clearly defined plan and endpoint in mind. As a team we combined efforts in researching and brainstorming to determine a general concept that our game would be based around. With a central idea defined, we each developed a paper prototype mock-up of what our version of the game would be and playtested it on people inexperienced with programming. The unified prototype was borne of our combined efforts and the information gathered from this initial research and development phase of \textit{Computron}.\\

After the unified prototype was formed, we each produced an identical whiteboard version of the game and tested it on as many people as we could. The testing went exceptionally well and we all agreed that we were ready to move forward with coding the game systems and mechanics. With a clear picture now forming as we began laying down the skeleton code to instantiate the basic layout of our game, we continued to fine tune the definitions and expectations of each aspect of our game and what the end result product should look like. This consisted not just of the physical appearance and feel of the game, but also the functionality of each component and how it should interact with other core components of the game. Additionally, the progression of puzzles that was designed for use with the prototype tested so well that it was adopted directly as the tutorial sequence for \textit{Computron}. Minor changes and additional levels were later added to the sequence, but the progression was found to be extremely effective.\\

Through frequent playtesting, we were able to refine a comprehensive design of our game. Working to implement key features was a driving focus, and informed the priority of minor features and supporting functionality. As a team, we strove to continuously keep our end goal in mind, and alwyas maintained the priority of each sprint to reflect the end goal of the project as a whole. Through this focus and determination, we are able to now deliver an exhaustive and satisfying first pass at \textit{Computron}.\\

\subsection{Future Considerations}
We realize that six months is an incredibly tight turnaround time for a fully realized video game project, and are well aware that there is potential for improvement and iteration on our final version of \textit{Computron}. It is our dearest hope that future students will take interest in the groundwork we have laid here and aspire to expand upon the current offerings and capabilities offered herein. To this end, we would like to present a summary of possible improvements to guide others towards success in their efforts.

As it currently stands, \textit{Computron} has only a basic implementation of tutorial sequences implemented. We would like to see future iterations of the project expand upon the number of provided tutorials. Keep in mind that a tutorial punctuating every level of the main path becomes more annoying than helpful, so it would also be beneficial to ensure there are enough buffer levels between newly introduced instructions or cards, and thus, tutorials. The frequency of introducing new game elements and interrupting the player to deliver instructions versus the progression of puzzles and challenge presented in solving each one is a delicate balance, and requires extensive testing and refinement. Proceed with caution.\\

In addition to expanding the tutorials available to players, it would behoove the next team to adopt this project to refine the tutorial construction system to be more user friendly. As it stands now, building a tutorial is a fairly involved process. In order to enable the streamlining of expansion on \textit{Computron} in any further iterations, it is integral to construct a tutorial building system that is intuitive and effortless. By streamlining this process, we can collectively open up the possibility of introducing \textit{Computron} as course software and allow professors to customize the puzzles provided to fit course materials.\\

In the same sense, it could be beneficial to the application of \textit{Computron} to lend greater support to WebGL builds from Unity. In its current state, the game experiences some scaling errors based on monitor resolution and whether or not the user is in full-screen mode. As we are currently only set to support deployment to an independent desktop build for the main three operating systems, including better web deployment stability could be largely beneficial. In addition to web-based access, a system that implements a database to keep track of leaderboard scores could be used in conjuction with Introduction to Programming classes in order to not only provide students with an incentive to practice programming concepts, but also to allow professors to track the progress of students enrolled in their courses. This insight could help lecturers focus on topics students struggle with and allow them to spend less time on topics that students handle well.\\

Additionally, there are some minor tweaks that could be made to enhance the player experience. First, the visuals of the game have some room for improvement. Currently there are animations missing on the Computron character. The artists we had on board were able to get sprite sheets into the game for animations but they were not implemented because the art style of the character and the puzzle scene contrasted. They were in the process of collaborating to transfer the sprite sheets to the puzzle scene art style, but unfortunately did not make it into the build in time. Also, in game objects like data cubes are extremely basic and could be much more impactful with a bit of polish. Second, more levels and card awarding can go a long way to improve player investment in \textit{Computron}. Currently, only newly acquired instructions are being distributed via the awarding system. Expanding this to include cards would be immensely beneficial to players in conjuction with the expanded tutorial system. More levels in the progression sequence could help space the time between tutorials -- as previously mentioned -- as well as increase the incentive for players to continue practicing their computational thinking skills. A complementary feature of this would be including tooltips on each instruction type and card type to remind players of the description of an element whenever it is appropriate to do so (e.g., on right click, double left click, etc.). Furthermore, future developers might aspire to construct a "card book" in the image of a card collector, where each card has an assigned slot in the book, and players can look through it at their leisure and review important information about each game element.\\

Some possible "nice to have" features for future iterations of \textit{Computron} might include deeply clever star requirements, allowing multiple solutions to be saved for each puzzle, and the opportunity to submit more efficient solutions to the developers. For current star requirements, players are simply measured against the metrics for the best solutions that the developers could come up. It would be nice for future teams to revisit those solutions and see if they can find more clever ways to get better ratings for the awards metrics. Similarly, if a player happens to get a better score than the metrics set so far, it would be beneficial to submit that solution to the development team so that the game could be updated to reflect a better solution. This would, of course, involve more iteration or explanation. Again, the idea of an online leaderboard comes to mind for claiming and reporting solution efficiency. Additionally, some metrics of efficiency versus instruction count versus cycle count may not all be attainable on a single pass at a puzzle solution. It would be advantageous to allow users to solve multiple solutions for each puzzle in the game. This way, an optimally time efficient solution and an optimally space efficient solution could both be stored for future reference.\\

\subsection{Conclusion}
This semester has been immensely successful for our team. By doing research and testing upfront to clearly define a vision for our game, we have eliminated the risk of coding ourselves into a corner and needing to start over or completely restructure the game system further into the life of the project. Additionally, the whiteboard version of our game is portable and easily testable with potential users. By running through the whiteboard version of our game with players from a variety of skill levels, we have been able to accumulate feedback that goes directly into informing the design of our game.\\

Our team has met and surpassed all of our project milestones to date, and we have a detailed plan laid out to keep us on track to preserve the progress we are making without compromising the quality or integrity of our project. A fully playable version of our game is due to be completed in March so that we can move the playtesting focus from the whiteboard version to the final computer-based version. The remaining phase of the project will consist mostly of fine-tuning and polishing based on the findings of the computer playtesting sessions.\\