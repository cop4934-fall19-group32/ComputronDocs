\textit{Computron} is a computer based puzzle-solving video game that is focused
on teaching basic Computer Science concepts and the process of computational
thinking to beginners in the field. There are currently many similar resources
available but they are either too technical for inexperienced programmers, or
 too abstract for players to apply what they learn to concepts in Computer
Science. \textit{Computron} bridges that gap by having players assemble code-like
solutions to puzzles in a simple pseudo-language. The solution is then executed
sequentially just like a computer program by an in-game character. This
execution is displayed to the player so that they can visually see the flow of
logic and adjust their solutions as needed. By beginning with the most basic
aspect of each concept and increasing the difficulty in a stepwise fashion,
\textit{Computron} is designed to help players acclimate to increasingly complex
algorithmic logic. Further, by providing feedback on solution efficiency,
\textit{Computron} also encourages players to improve upon the performance of
their solution.\\

The main objective of this project is to help individuals who are approaching
Computer Science for the first time by introducing them to basic concepts and
encouraging them to think programmatically. The layout of the game tutorial
and slow but steady increase in puzzle difficulty is specifically designed to coach
players through the concepts to ensure that they understand the basic aspects
of each concept and how to implement it in a coding environment to manipulate
data. This also supports one of the secondary objectives, which is to encourage
more people to pursue their interests in Computer Science without feeling
overwhelmed with everything that beginners typically need to learn.\\

\textit{Computron} is designed with Unity. The development and processing of 
a simple pseudo-language was needed in order to cover the types of operations 
users would be expected to execute in order to solve puzzles. The puzzles and the 
processes needed to solve them were assembled around basic concepts in Computer 
Science. The progression of difficulty and determining when each aspect would be 
introduced are based off of course outlines from various beginner level courses, 
including Intro to C and Computer Science I. Thorough research and testing has 
gone into the decisions behind these processes and defining the scope of the project.\\