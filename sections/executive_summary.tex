\textit{Computron} is a computer based puzzle-solving video game that is focused on teaching basic computer science concepts and the process of computational thinking to beginners. There are currently many similar resources available, but they are either too technical for inexperienced programmers, or too abstract for players to apply what they learn to concepts in computer science. \textit{Computron} bridges that gap by having players assemble code-like solutions to puzzles in a simple pseudo-language. The solution is then executed programmatically by an in-game character. This execution is displayed to the player so that they can see the flow of logic and adjust their solutions as needed. By beginning with the most basic aspect of each concept and increasing the difficulty in a stepwise fashion, \textit{Computron} is designed to help players acclimate to increasingly complex algorithmic logic. Further, by providing feedback on solution efficiency, \textit{Computron} also encourages players to improve upon the performance of their solution.\\

The main objective of this project is to help individuals who are approaching computer science for the first time by introducing them to basic concepts and encouraging them to think programmatically. The layout of the game tutorial and slow but steady increase in puzzle difficulty is specifically designed to coach players through the concepts to ensure that they understand the basic aspects of each concept. This also supports one of the secondary objectives, which is to encourage more people to pursue their interests in computer science without feeling overwhelmed with everything that beginners typically need to learn.\\

\textit{Computron} is developed in Unity, and is playable on Windows, Mac, and Linux. The core gameplay loop has players interact with the Computron Assembly Language (CAL) and a selection of datastructures to solve puzzles. The CAL is a collection of 13 assembly-like instructions that allow players to manipulate the elements of the puzzles. The flow of the puzzles was designed with basic concepts in computer science in mind, such as sorting, logical comparisons, and arithmetic. The progression of puzzle difficulty and the order in which each game mechanic would be introduced are based off of course outlines from various beginner level courses, including Introduction to C and Computer Science I. Thorough research and testing has gone into the decisions behind these processes and defining the scope of the project.